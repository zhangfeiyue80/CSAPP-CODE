\documentclass{article}
\usepackage{CJKutf8}
\usepackage{color}
\usepackage{array}
\usepackage[margin=1cm,top=1cm,bottom=1.5cm]{geometry}
\newcommand{\blue}[1]{\textcolor{blue}{#1}}
\newcommand{\red}[1]{\textcolor{red}{#1}}
\begin{document}
\begin{CJK}{UTF8}{gkai}
\end{CJK}
\begin{CJK}{UTF8}{gbsn}
\textcolor[rgb]{1,0,0}{\begin{math} \end{math} \begin{math}			\end{math}}
\begin{math}			
	\vec{L}=\vec{r}\times m \vec{v}
\end{math}\\[0.1cm]

\begin{math}
	\frac{d \vec{L}}{dt}
	=\frac{d \vec{r}}{dt}\times{}m \vec{v}+\vec{r}\times{}m \frac{d \vec{v}}{dt}
	=\vec{v}\times{}m \vec{v}+\vec{r}\times{}m \vec{a}
	=\vec{r}\times{}m \vec{a}
	=\vec{r}\times{}\vec{F}
\end{math}\\[0.1cm]
\begin{math}	
	\lceil x/y\rceil=\lfloor (x+y-1)/y\rfloor	\\
	\lceil x/y\rceil-1<x/y\le \lceil x/y\rceil	\\
	\lfloor (x+y-1)/y \rfloor\le (x+y-1)/y<\lfloor (x+y-1)/y \rfloor+1	\\
\end{math}
证明: $x/y+1-1/y>\lceil x/y\rceil$	\\
$\left( x/y-\lceil x/y\rceil+1 \right)_{min}=(x/y-\lfloor x/y\rfloor)_{min}=1/y$	\\
$x/y-1/y+1\ge\lfloor x/y\rfloor+1=\lceil x/y\rceil$	\\
$x/y-1/y+1\ge\lfloor x/y-1/y+1\rfloor$	\\
$\lceil x/y\rceil=\lfloor x/y-1/y+1\rfloor$	\\[3ex]
x/16	\\
$x\ge0$	,$x>>4$	\\
$x<0$,$(x+(1<<4)-1)>>4$	\\
posi\_neg(x),正数返回0,负数返回1	\\
$x/16=(x+((1<<4)-1)*post\_neg(x))>>4=(x+((1<<4)-1)\&(x>>32))>>4$	\\
$x*0=x<<32$	\\
$x*1=x>>0$
位与运算
位或运算	\\[3ex]
\begin{math}
	n!=n*(n-1)*(n-2)*\cdots*3*2*1	\\
	(n!)^{'}=(n-1)*(n-2)*\cdots*3*2*1+n\left( (n-1)*(n-2)*\cdots*3*2*1 \right )^{'}	\\
	2^n=n*2^{n-1}
\end{math}	\\[3ex]
\begin{math}
	E=e-bias	\\
	bias=2^{k-1}-1	\\
	E=e-2^{k-1}+1	\\
	=2^{k-1}+2^{k-1}-2-2^{k-1}+1	\\
	=2^{k-1}-1
\end{math}	\\
$0111.1_2=1.111_{2}\times 2^2$	\\
bias=7	\\
1001	\\
1001 111	\\[2ex]
$M=1+2^(-1)+2^(-4)=25/16$	\\
$E=2-3=-1$	\\
$\frac{25}{16}*2^{-1}=\frac{25}{32}$	\\
$25/32=0.11001_2=1.1001_2*2^{-1}$	\\
$0110\ 1001_2\approx 0110\ 100_2$	\\
M=3/2	\\
E=6-7=-1	\\
$\frac{3}{2}*2^{-1}=\frac{3}{4}$	\\
$M=1+2^{-1}+2^{-2}+2^{-3}+2^{-4}=1-2^{-4}=\frac{15}{16}$	\\
$E=6-3=3$	\\
$\frac{15}{16}*2^3=\frac{15}{2}$	\\
$111.1_2=1.111*2^2$	\\
1001 111,M=1/16,E=1-3=-2	\\
$\frac{1}{16}*2^{-2}=\frac{1}{64}$	\\
$0.000001=1.0*2^{-6}$	\\
0001 000	\\
$x=a_{31}a_{30}a_{29}a_{28}a_{27}a_{26}a_{25}a_{24}a_{23}a_{22}a_{21}a_{20}a_{19}a_{18}a_{17}a_{16}a_{15}a_{14}a_{13}a_{12}a_{11}a_{10}a_{9}a_{8}a_{7}a_{6}a_{5}a_{4}a_{3}a_{2}a_{1}a_{0}$	\\
$x=a_{31}.a_{30}a_{29}a_{28}a_{27}a_{26}a_{25}a_{24}a_{23}a_{22}a_{21}a_{20}a_{19}a_{18}a_{17}a_{16}a_{15}a_{14}a_{13}a_{12}a_{11}a_{10}a_{9}a_{8}a_{7}a_{6}a_{5}a_{4}a_{3}a_{2}a_{1}a_{0}*2^{31}$	\\
$1.0\times10^{ 20 }$	\\
%	45w
%	120w
%	ec 78 ad 60
%1	110 1100 0	111 1000 1010 1101 0110 0000
%1	110 1100 0000	111 1000 1010 1101 0110 0000  0000 0000 0000 0000 0000 0000 0000 0
%	100000002004087734272.000000
%	00 00 c8 42
%	0000 0000 0000 0000 1100 1000 0100 0010
%	0	1000 0100 	0010 0110 0000 0000 0000 000
%	1*10^(128+4-127)=10 0000
%	0 1000 0101	 1001 0000 0000 0000 0000 000
%	1.1001*10^(128-127+5)
%	0 	0111 1111 	0000 0000 0000 0000 0000 000
%	0011 1111 1000 0000 0000 0000 0000 0000
%	00 00 80 3f
%	3f 80 00 00
%	0 	0111 1111 	0000 0000 0000 0000 0000 000
%	ec 78 ad 60
%	60 ad 78 ec
%	0110 0000 1010 1101 0111 1000 1110 1100



%	0	1100 0001	 0101 1010 1111 0001 1101 100
%	1*2^(128-127+64+1)=2^66

%	0	100 0100 0001	0101 1010 1111 0001 1101 1000 0000 0000 0000 0000 0000 0000 0000
%	0	011 1111 1111	0000 0000 0000 0000 0000 0000 0000 0000 0000 0000 0000 0000 0000
%	0	000 0100 0000



%	00 00 00 80 1d af 15 44

%	44 15 af 1d 80 00 00 00
%	44 15 af 1d 80 00 00 00

%	0100 0100 0001 0101 1010 1111 0001 1101 1000 0000 0000 0000 0000 0000 0000 0000
%	0	100 0100 0001 	0101 1010 1111 0001 1101 1000 0000 0000 0000 0000 0000 0000 0000

%	1.0101 1010 1111 0001 1101 1000 0000 0000 0000 0000 0000 0000 0000*2^66
%	9+0x100+0x3=0x10C
%	260+0x1+0x3=0x104+0x1+0x3=0x108
%	0xFC+0x1*4=0xFC+0x4=0x100
%	0x100+0x3*4=0x100+0xC=0x10C
%	0x1+0xFF=0x100
%	0x108
%	0xAB-0x3=0xA8
%	0x3
%	0x100+0x3*8=0x100+0x18=0x118
%	0x11*16=0x10+0x10*16=0x10+0x100=0x110
%	0x100+0x10=0x110
%	0x1
%	0x100-0x3=0xFD

%	11100
%	11000
%	00100
%	400431+f4=400425
%	ff 73
%	ff 73
%	4005ed
%	4

%	2016
%	f0
%	2006

%	2016
%	70
%	2086

%1111 0100
%1 0100
%-16+4=-12
%0400431
%ffffff4
%0400425

%ff ff ff 73
%00 40 05 ed
%00 40 05 60
%\noindent\begin{table}[htbp]
\newpage
\textbf{练习题3.21分析}	\\
\begin{table}[ht]
\begin{tabular}{m{1em}m{3em}m{12em}l}
		&	\multicolumn{2}{l}{\blue{long test(long x, long y)}}	&	\\
		&	\multicolumn{2}{l}{\blue{x in \%rdi, y in \%rsi}}	&	\\
	test:	\\
	&	leaq	&	0(,\%rdi,8), \%rax				&	\blue{\%rax=8*x}	\\
	&	testq	&	\%rsi, \%rsi					&	\blue{y \& y}	\\
		&	jle	&	.L2					&	\blue{y $<=$ 0, goto .L2}\\
		&	movq	&	\%rsi, \%rax				&	\blue{\%rsi $>$ 0, \%rax = y}\\
		&	subq	&	\%rdi, \%rax				&	\blue{\%rax = y - x}\\
		&	movq	&	\%rdi, \%rdx				&	\blue{\%rdx = x}\\
		&	andq	&	\%rsi, \%rdx				&	\blue{\%rdx = x \& y}\\
		&	cmpq	&	\%rsi, \%rdi				&	\blue{x - y}	\\
		&	cmovge	&	\%rdx, \%rax				&	\blue{(x -y) $>=$ 0, \%rax = x \& y}	\\
		&	ret	&						&	\blue{return \%rax}	\\
	.L2:	\\
	&	addq	&	\%rsi, \%rdi					&	\blue{\%rdi = x + y}	\\
	&	cmpq	&	\$-2, \%rsi					&	\blue{y + 2}\\
	&	cmovle	&	\%rdi, \%rax					&	\blue{(y + 2) $<=$ 0, \%rax = x + y}	\\
	&	ret	&							&	\blue{return \%rax}	\\
\end{tabular}
\end{table}	\\
\textbf{练习题3.23分析}	\\
\begin{table}[ht]
\begin{tabular}{m{1em}m{3em}m{12em}l}
		&	\multicolumn{2}{l}{\blue{long dw\_loop(long x)}}	&	\\
		&	\multicolumn{2}{l}{\blue{x initially in \%rdi}}	&	\\
	dw\_loop:	\\
	&	movq	&	\%rdi, \%rax	&	\blue{\%rax = x}	\\
	&	movq	&	\%rdi, \%rcx	&	\blue{\%rcx = x}	\\
	&	imulq	&	\%rdi, \%rcx	&	\blue{\%rcx = x*x}	\\
	&	leaq	&	(\%rdi, \%rdi), \%rdx	&	\blue{\%rdx = 2*x}	\\
	.L2:	\\
	&	leaq	&	1(\%rcx, \%rax), \%rax	&	\blue{\%rax = 1 + $x^2$ + x}	\\
	&	subq	&	\$1, \%rdx		&	\blue{\%rdx = 2*x - 1}	\\
	&	testq	&	\%rdx, \%rdx		&	\blue{( 2*x-1 ) \& ( 2*x-1 )}	\\
	&	jg	&	.L2			&	\blue{( 2*x-1 )$>$0, goto .L2; ( 2*x-1 )$<=0$; return \%rax }	\\
	&	rep;	&	ret			&	\blue{}	\\
\end{tabular}
\end{table}	\\
% %rdx = %rdi+b
% %rax = %rax * ( %rdi+b )
% %rdi = %rdi+1

% %rax = %rax*a
% %rsi = %rsi - a
% while( %rsi>0 )

%while(x != 0)
%{
%	val = val ^ x;
%	x = x >> 1;
%}
%val = val & 1;
%return val;
\newpage
\noindent\textbf{练习题3.27分析}	\\[-3ex]
\begin{table}[ht]
\begin{tabular}{m{2em}m{2em}m{2em}l}
	\multicolumn{3}{l}{long fact\_for\_guarded\_do(long n)}	\\
\{	\\
	&	\multicolumn{3}{l}{long i = 2;}	\\
	&	\multicolumn{3}{l}{long result = 1;}	\\
	&	\multicolumn{3}{l}{ if (i $>$ n)  } 	\\
	&	&	\multicolumn{2}{l}{goto done;} 	\\
	\multicolumn{3}{l}{loop:} 	\\
	&	&	\multicolumn{2}{l}{result *= i;}	\\
	&	&	\multicolumn{2}{l}{i++;}	\\
	&	\multicolumn{3}{l}{if (i $<=$ n)}	\\
	&	&	\multicolumn{2}{l}{goto loop;} 	\\
	\multicolumn{3}{l}{done:} 	\\
	&	\multicolumn{3}{l}{return result;} 	\\
\}	\\
\end{tabular}
\end{table}	\\
\textbf{练习题3.28分析}	\\[-3ex]
\begin{table}[ht]
\begin{tabular}{m{2em}m{2em}m{8em}l}
	\multicolumn{3}{l}{\blue{ long fun\_b(unsigned long x) }}	\\
	\multicolumn{3}{l}{\blue{ x in \%rdi, i in \%rdx }}	\\
	\multicolumn{3}{l}{fun\_b:}	\\
	&	movl	&	\$64, \%edx 	&	\blue{i = 64}	\\
	&	movl	&	\$0, \%eax	&	\blue{val = 0}	\\
	\multicolumn{3}{l}{.L10:}	\\
	&	movq	&	\%rdi, \%rcx 	&	\blue{\%rcx = x}	\\
	&	andl	&	\$1, \%ecx 	&	\blue{x is odd ,\%ecx = 1 ; x is even,\%ecx = 0}	\\
	&	addq	&	\%rax, \%rax 	&	\blue{val = 2*val}	\\
	&	orq	&	\%rcx, \%rax 	&	\blue{val = \%rcx $|$ val}	\\
	&	shrq	&	\%rdi 	&	\blue{ Shift right by 1; x = x / 2 }	\\
	&	subq	&	\$1, \%rdx 	&	\blue{i = i - 1}	\\
	&	jne	&	.L10 	&	\blue{if i != 0,goto .L10}	\\
	&	rep;	&	ret 	&	\blue{return \%rax}	\\
\end{tabular}
\end{table}	\\
% %rcx=x&1
% val=2*val
% val=(x&1)|(2*val)
% x=x/2
% i--

% i	64,		63,		62,
% val	x1(二进制末位),	2*x1|x2,	2*(2*x1|x2)|x3
% 0011010001
%   0100111 

% 0010000000
% 1

% 0010000001
% 0

% 0010000011
% 1
\noindent\textbf{练习题3.29分析}	\\
\textbf{A. }	\\[-3ex]
\begin{table}[ht]
\begin{tabular}{m{2em}m{2em}m{2em}l}
	\multicolumn{3}{l}{long sum = 0;}	\\
	\multicolumn{3}{l}{long i;}	\\
	\multicolumn{3}{l}{i = 0;}	\\
	\multicolumn{3}{l}{while(i $<$ 10)}	\\
	\multicolumn{3}{l}{\{}	\\
	&	\multicolumn{2}{l}{if(i \& 1)}	\\
	&	&	continue;	\\
	&	\multicolumn{2}{l}{sum += 1;}	\\
	&	\multicolumn{2}{l}{i++;}	\\
	\multicolumn{3}{l}{\}}	\\
\end{tabular}
\end{table}	\\

\newpage
\noindent\textbf{练习题3.30分析}	\\[-4ex]
\begin{table}[ht]
\begin{tabular}{m{2em}m{2em}m{8em}l}
	\multicolumn{3}{l}{\blue{ void switch2(long x,long *dest) }}	\\
	\multicolumn{3}{l}{\blue{x in \%rdi}}	\\
	\multicolumn{3}{l}{switch2:}	\\
	&	addq	&	\$1, \%rdi	&	\blue{ x=x+1 }	\\
	&	cmpq	&	\$8, \%rdi	&	\blue{x-8}	\\
	&	ja	&	.L2		&	\blue{\blue{x-8$>$0, 无符号大于,goto .L2}}	\\
	&	jmp	&	*.L4(,\%rdi,8)	&	\blue{goto .L4+8x}	\\
\end{tabular}
\end{table}	\\[-2ex]
\textbf{练习题3.31分析}	\\[-4ex]
\begin{table}[ht]
\begin{tabular}{m{2em}m{3em}m{8em}l}
	\multicolumn{3}{l}{\blue{void switcher(long a, long b, long c, long *dest)}}	\\
	\multicolumn{3}{l}{\blue{a in \%rdi, b in \%rsi, c in \%rdx, dest in \%rcx}}	\\
	\multicolumn{3}{l}{switcher:}	\\
	&	cmpq		&	\$7, \%rdi	&	\blue{a -- 7}	\\
	&	ja		&	.L2		&	\blue{a -- 7 $>$ 0, unsigned a $>$ 7, long a $<$ 0 or a $>$ 7}	\\
	&	jmp		&	*.L4(,\%rdi,8)	&	\blue{goto *.L4+8*a}	\\
	&	.section	&	.rodata		&	\blue{}	\\
.L7:	&			&			&	\blue{}	\\
&	xorq		&	\$15, \%rsi	&	\blue{b = b\^{}15}	\\
	&	movq		&	\%rsi, \%rdx	&	\blue{c = b}	\\
.L3:	&			&			&	\blue{}	\\
	&	leaq		&	112(\%rdx), \%rdi		&	\blue{a = 112 + c}	\\
	&	jmp		&	.L6		&	\blue{goto .L6}	\\
.L5:	&			&			&	\blue{}	\\
	&	leaq		&	(\%rdx,\%rsi), \%rdi		&	\blue{a = c + b}	\\
	&	salq		&	\$2, \%rdi	&	\blue{a = a*4}	\\
	&	jmp		&	.L6		&	\blue{goto .L6}	\\
.L2:	&			&			&	\blue{}	\\
	&	movq		&	\%rsi, \%rdi	&	\blue{a = b}	\\
.L6:	&			&			&	\blue{}	\\
	&	movq		&	\%rdi, (\%rcx)	&	\blue{*dest = a}	\\
	&	ret		&			&	\blue{return \%rax}	\\
\end{tabular}
\end{table}	\\[-2ex]
\begin{table}[ht]
\begin{tabular}{m{2em}m{3em}m{8em}l}
.L4:	\\
	&	\multicolumn{2}{l}{.quad}	&	.L3	\\
	&	\multicolumn{2}{l}{.quad}	&	.L2	\\
	&	\multicolumn{2}{l}{.quad}	&	.L5	\\
	&	\multicolumn{2}{l}{.quad}	&	.L2	\\
	&	\multicolumn{2}{l}{.quad}	&	.L6	\\
	&	\multicolumn{2}{l}{.quad}	&	.L7	\\
	&	\multicolumn{2}{l}{.quad}	&	.L2	\\
	&	\multicolumn{2}{l}{.quad}	&	.L5	\\
\end{tabular}
\end{table}	\\[-2ex]
.L2 1,3,6	\\
.L5 2,7		\\
\red{ 0 },\red{ 1 },\red{ 2 },\red{ 3 },4,\red{ 5 },\red{ 6 },\red{ 7 }
\newpage
\begin{table}[ht]
\begin{tabular}{m{2em}m{3em}m{8em}l}
.L4:	\\
	&	\multicolumn{2}{l}{.quad}	&	.L2	\\
	&	\multicolumn{2}{l}{.quad}	&	.L2	\\
	&	\multicolumn{2}{l}{.quad}	&	.L2	\\
	&	\multicolumn{2}{l}{.quad}	&	.L3	\\
	&	\multicolumn{2}{l}{.quad}	&	.L5	\\
	&	\multicolumn{2}{l}{.quad}	&	.L5	\\
	&	\multicolumn{2}{l}{.quad}	&	.L6	\\
	&	\multicolumn{2}{l}{.quad}	&	.L7	\\
\end{tabular}
\end{table}	
\textbf{练习题3.35分析}	\\
\blue{long rfun(unsinged long x)}	\\
\blue{x in \%rdi}	\\
rfun:	\\
	\hspace*{1em}pushq \%rbx	\\
	\hspace*{1em}movq \%rdi, \%rbx	\blue{\%rbx = x}\\
	\hspace*{1em}movl \$0, \%eax	\blue{\%eax = 0}\\
	\hspace*{1em}testq \%rdi, \%rdi	\blue{x \& x}\\
	\hspace*{1em}je .L2	\blue{x ==  0 goto .L2}\\
	\hspace*{1em}shrq \$2, \%rdi	\blue{x / 4}\\
	\hspace*{1em}call rfun	\\
	\hspace*{1em}addq \%rbx, \%rax	\\
.L2:	\\
	\hspace*{1em}popq \%rbx	\\
	\hspace*{1em}ret	\\
\textbf{练习题3.38分析}	\\
\blue{long sum\_element(long i, long j)}	\\
\blue{i in \%rdi, j in \%rsi}	\\
sum\_element:	\\
\hspace*{1em}leaq 0(, \%rdi,8), \%rdx	\blue{\%rdx = 8i}	\\
\hspace*{1em}subq \%rdi, \%rdx	\blue{\%rdx = 8i-i = 7i}	\\
\hspace*{1em}addq \%rsi, \%rdx	\blue{\%rdx = 7i + j}	\\
\hspace*{1em}leaq (\%rsi,\%rsi,4), \%rax	\blue{\%rax = j+4j=5j}	\\
\hspace*{1em}addq \%rax, \%rdi	\blue{\%rdi = i+5j}	\\
\hspace*{1em}movq Q(,\%rdi,8), \%rax	\blue{\%rax = M[8i+40j+Q]}	\\
\hspace*{1em}addq P(,\%rdx,8), \%rax	\blue{\%rax = M[8i+40j+Q]+M[56i+8j+P]}	\red{P[i][j]=M[P+8(Ni+j)],Q[j][i]=M[Q+8(Mj+i)]}	\\
\hspace*{1em}ret	\blue{return \%rax}	\\[2ex]
\textbf{练习题3.40分析}	\\[1ex]
fix\_set\_diag:		\\
\blue{void fix\_set\_diag(fix\_matrix A, int val)}	\\
\blue{A in \%rdi, val in \%rsi}	\\
	\hspace*{1em}movl \$0, \%eax	\blue{\%eax = 0}	\\
.L13:	\\
	\hspace*{1em}movl \%esi, (\%rdi,\%rax)	\blue{M[A+0] = val}	\\
	\hspace*{1em}addq \$68, \%rax	\blue{\%rax = \%rax+68}	\\
	\hspace*{1em}cmpq \$1088, \%rax	\blue{compare 1088 and \%rax}	\\
	\hspace*{1em}jne .L13	\blue{\%rax != 1088 , goto L13}	\\
	\hspace*{1em}rep; ret	\\[2ex]
A[i][i]=M[A+( Ni+i )]	\\
A+(Ni+i)=A+i(N+1)	\\
A+(i+1)(N+1)-A-i(N+1)=(i+1)(N+1)-i(N+1)=iN+i+N+1-(iN+i)=N+1	\\[2ex]
\newpage
\noindent\textbf{练习题3.41分析}	\\
\blue{void sp\_init(struct prob *sp)}	\\
\blue{sp in \%rdi}	\\
sp\_init:	\\
	\hspace*{1em}movl	12(\%rdi), \%eax	\blue{\%eax = M[sp + 12]}	\\
	\hspace*{1em}movl	\%eax, 8(\%rdi)		\blue{M[sp + 8] = M[sp + 12]}	\\
	\hspace*{1em}leaq	8(\%rdi), \%rax		\blue{\%rax = sp + 8}	\\
	\hspace*{1em}movq	\%rax, (\%rdi)		\blue{M[sp] = sp + 8}	\\
	\hspace*{1em}movq	\%rdi, 16(\%rdi)	\blue{M[sp + 16] = sp}	\\
	\hspace*{1em}ret	\blue{return}	\\[2ex]
\textbf{练习题3.42分析}	\\
\blue{long fun(struct ELE *ptr)}	\\
\blue{ptr in \%rdi}	\\
fun:	\\
	\hspace*{1em}movl	\$0, \%eax	\blue{\%eax = 0}	\\
	\hspace*{1em}jmp	.L2		\blue{goto .L2}		\\
.L3:	\\
	\hspace*{1em}addq	(\%rdi), \%rax	\blue{\%rax = \%rax + M[ptr]}	\red{\%rax = \%rax + ptr$->$v}	\\
	\hspace*{1em}movq	8(\%rdi), \%rdi	\blue{\%rdi = M[ptr + 8]}	\red{ptr = ptr$->$p}	\\
.L2	\\
	\hspace*{1em}testq	\%rdi, \%rdi	\blue{\%rdi \& \%rdi}	\red{ptr \& ptr}	\\
	\hspace*{1em}jne	.L3	\blue{\%rdi != 0,goto .L3}	\red{ptr != 0, goto .L3}	\\
	\hspace*{1em}rep; ret		\blue{return}	\\[3ex]
\textbf{练习题3.50分析}	\\[1ex]
\blue{double fcvt2(int *ip, float *fp, double *dp, long l)}\\
\blue{ip in \%rdi, fp in \%rsi, dp in \%rdx, l in \%rcx}	\\
\blue{Result returned in \%xmm()}	\\
fcvt2:	\\
	\hspace*{1em}movl (\%rdi), \%eax \blue{\%eax=*ip}	\\
	\hspace*{1em}vmovss (\%rsi), \%xmm0 \blue{\%xmm0=*fp}	\\
	\hspace*{1em}vcvttsd2si (\%rdx) \%r8d \blue{\%r8d=(int) *dp}	\\
	\hspace*{1em}movl \%r8d, (\%rdi) \blue{*ip=(int) *dp}	\\
	\hspace*{1em}vcvtsi2ss \%eax, \%xmm1, \%xmm1 \blue{\%xmm1=(float) *ip}	\\
	\hspace*{1em}vmovss \%xmm1, (\%rsi) \blue{*fp=(float) *ip}	\\
	\hspace*{1em}vcvtsi2sdq \%rcx, \%xmm1, \%xmm1 \blue{\%xmm1=(double) l}	\\
	\hspace*{1em}vmovsd \%xmm1, (\%rdx) \blue{*dp=(double) l}	\\
	\hspace*{1em}vunpcklps \%xmm0, \%xmm0, \%xmm0 \blue{}	\\
	\hspace*{1em}vcvtps2pd \%xmm0, \%xmm0 \blue{return (double) *fp}	\\
	\hspace*{1em}ret	\\[3ex]
\textbf{练习题3.53分析}	\\
\blue{double funct1(arg1\_t p, arg2\_t q,arg3\_t r, arg4\_t s)}	\\
funct1:	\\
\hspace*{1em}vcvtsi2ssq	\%rsi, \%xmm2, \%xmm2 \blue{$\%xmm2_{float}=\%rsi_{long}$}	\\
\hspace*{1em}vaddss	\%xmm0, \%xmm2, \%xmm0 \blue{$\%xmm0_{float}=\%xmm0_{float}+\%xmm2_{float}$}	\\
\hspace*{1em}vcvtsi2ss	\%rdi, \%xmm2, \%xmm2 \blue{$\%xmm2_{float}=\%rdi_{int}$}	\\
\hspace*{1em}vdivss	\%xmm0, \%xmm2, \%xmm0 \blue{$\%xmm0_{float}=\%xmm2_{float}/\%xmm0_{float}$}	\\
\hspace*{1em}vunpcklps	\%xmm0, \%xmm0, \%xmm0 \blue{}	\\
\hspace*{1em}vcvtps2pd	\%xmm0, \%xmm0 \blue{$\%xmm0_{double}=(double)\%xmm0_{float}$}	\\
\hspace*{1em}vsubsd	\%xmm1, \%xmm0, \%xmm0 \blue{$\%xmm0_{double}=\%xmm0_{double}-\%xmm1_{double}$}	\\
\hspace*{1em}ret \blue{return $\%xmm0_{double}$}	\\[3ex]
\newpage
\noindent\textbf{练习题3.54分析}	\\
\blue{double funct2(double w, int x, float y, long z)}	\\
\blue{w in \%xmm0, x in \%edi, y in \%xmm1, z in \%rsi}	\\
funct2:	\\
\hspace*{1em}vcvtsi2ss	\%edi, \%xmm2, \%xmm2 \blue{$\%xmm2_{float}=(float)x_{int}$}	\\
\hspace*{1em}vmulss	\%xmm1, \%xmm2, \%xmm1 \blue{$y_{float}=y_{float}*(float)x_{int}$}	\\
\hspace*{1em}vunpcklps	\%xmm1, \%xmm1, \%xmm1 \blue{}	\\
\hspace*{1em}vcvtps2pd	\%xmm1, \%xmm2 \blue{$\%xmm2_{double}=(double)y_{float}=(double)(y_{float}*(float)x_{int})$}	\\
\hspace*{1em}vcvtsi2sdq	\%rsi, \%xmm1, \%xmm1 \blue{$y_{double}=(double)z_{long}$}	\\
\hspace*{1em}vdivsd	\%xmm1, \%xmm0, \%xmm0 \blue{$w_{double}=w_{double}/(double)z_{long}$}	\\
\hspace*{1em}vsubsd	\%xmm0, \%xmm2, \%xmm0 \blue{$w_{double}=(double)(y_{float}*(float)x_{int})-w_{double}/(double)z_{long}$}	\\
\hspace*{1em}ret \blue{return $w_{double}$}	\\[3ex]
\textbf{练习题3.56分析}	\\
\textbf{A.}	\\
	\hspace*{1em}vmovsd	.LC1(\%rip), \%xmm1	\\
	\hspace*{1em}vandpd	\%xmm1, \%xmm0, \%xmm0 \blue{x \& 0x7f ff ff ff ff ff ff ff}	\\
.LC1:	\\
	\hspace*{1em}.long	4294967295 \blue{低位4个字节:0xff ff ff ff}	\\
	\hspace*{1em}.long	2147483647 \blue{高位4个字节:0x7f ff ff ff}	\\
	\hspace*{1em}.long	0	\\
	\hspace*{1em}.long	0	\\
\textbf{B.}	\\
	\hspace*{1em}vxorpd	\%xmm0, \%xmm0, \%xmm0 \blue{\%xmm0=0x0}	\\
\textbf{C.}	\\
	\hspace*{1em}vmovsd	.LC2(\%rip), \%xmm1 \blue{}	\\
	\hspace*{1em}vxorpd	\%xmm1, \%xmm0, \%xmm0 \blue{0x80 00 00 00 00 00 00 00 \& x}	\\
.LC2:	\\
	\hspace*{1em}.long	0 \blue{0x0}	\\
	\hspace*{1em}.long	-2147483648 \blue{0x80 00 00 00}	\\
	\hspace*{1em}.long	0 \blue{0x0}	\\
	\hspace*{1em}.long	0 \blue{0x0}	\\[3ex]
\textbf{练习题3.57分析}	\\
\blue{double funct3(int *ap, double b, long c, float *dp)}	\\
\blue{ap in \%rdi, b in \%xmm0, c in \%rsi, dp in \%rdx}	\\
funct3:	\\
	\hspace*{1em}vmovss	(\%rdx), \%xmm1 \blue{$\%xmm1_{float}=*dp_{float}$}	\\
	\hspace*{1em}vcvtsi2sd	(\%rdi), \%xmm2, \%xmm2 \blue{$\%xmm2_{double}=(double)*ap$}	\\
	\hspace*{1em}vucomisd	\%xmm2, \%xmm0 \blue{compare $b_{double}:(double)*ap$}	\\
	\hspace*{1em}jbe	.L8 \blue{$b_{double}\le(double)*ap$, goto .L8}	\\
	\hspace*{1em}vcvtsi2ssq	\%rsi, \%xmm0, \%xmm0 \blue{$\%xmm0_{float}=(float)c_{long}$}	\\
	\hspace*{1em}vmulss	\%xmm1, \%xmm0, \%xmm1 \blue{$\%xmm1_{float}=*dp_{float}*(float)c_{long}$}	\\
	\hspace*{1em}vunpcklps	\%xmm1, \%xmm1, \%xmm1 \blue{}	\\
	\hspace*{1em}vcvtps2pd	\%xmm1, \%xmm0 \blue{$\%xmm0_{double}=(double)\%xmm1_{float}$}	\\
	\hspace*{1em}ret \blue{return \%xmm0}	\\
.L8:	\\
\hspace*{1em}vaddss	\%xmm1, \%xmm1, \%xmm1 \blue{$*dp_{float}=*dp_{float}+*dp_{float}$}	\\
	\hspace*{1em}vcvtsi2ssq	\%rsi, \%xmm0, \%xmm0 \blue{$\%xmm0_{float}=(float)c_{long}$}	\\
	\hspace*{1em}vaddss	\%xmm1, \%xmm0, \%xmm0 \blue{$\%xmm0_{float}=*dp_{float}+(float)c_{long}$}	\\
	\hspace*{1em}vunpcklps	\%xmm0, \%xmm0, \%xmm0 \blue{}	\\
	\hspace*{1em}vcvtps2pd	\%xmm0, \%xmm0 \blue{$\%xmm0_{double}=(double)\%xmm0_{float}$}	\\
	\hspace*{1em}ret \blue{return $\%xmm0_{double}$}	\\



\end{CJK}
\end{document}
