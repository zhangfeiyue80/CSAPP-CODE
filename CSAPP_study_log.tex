\documentclass{article}
\usepackage{CJK}
\usepackage{type1cm}
\usepackage{color}
\usepackage{array}
\usepackage[margin=1cm,top=1cm,bottom=2cm]{geometry}
\newcommand{\red}[1]{\textcolor{red}{#1}}
\begin{document}
\begin{CJK}{UTF8}{gkai}
\end{CJK}
\begin{CJK}{UTF8}{gbsn}
\end{CJK}
\begin{CJK}{UTF8}{uyhei}
%\textcolor[rgb]{1,0,0}{\begin{math} \end{math} \begin{math}			\end{math}}
%\centerline{\Huge{CSAPP Study Log}}		\\[3ex]
\fontsize{40pt}{40pt}\selectfont
\centerline{CSAPP Study Log}			
\fontsize{25pt}{25pt}\selectfont
\centerline{\red{refinement:右脑阅读,图像阅读}}	
\fontsize{18pt}{18pt}\selectfont
\noindent\textbf{1)}			\\
$_w^u+x$无符号加法			\\[0.5ex]
表示与x相加后,取w位无符号整数		\\[0.5ex]
$_w^t+x$补码加法			\\[0.5ex]
$-_w^tx$表示x在$^t_w+$下的加法逆元	\\[0.5ex]
$-_w^ux$表示x在$_w^u+$下的加法逆元	\\[1ex]
\textbf{2)}
``试图最大化一段关键代码性能的程序员,通常会尝试源代码的各种形式''包括汇编	\\[1ex]
\textbf{3)}
精通细节是理解更深和更基本概念的先决条件	\\[1ex]
\textbf{4)}	\\
栈	stacks	\\[1ex]
栈指针	\%rsp	(stack pointer)	\\[1ex]
\textbf{5)}	\\
SAL(shift arithmetic left)	算术左移	\\
SHL(shift logical left)		逻辑左移	\\
SAR(shift arithmetic right)	算术右移	\\
SHR(shift logical right)	逻辑右移	\\[1ex]
\textbf{6)}	\\
补码	two's complement	\\
\textbf{7)}	\\
处理器通过使用流水线(pipelining)来获得高性能,在流水线中,一条指令的处理要经过一系列的阶段,每个阶段执行所需操作的一小部分(例如,从内存取指令、确定指令类型、从内存读数据、执行算术去处、向内存写数据,以及更新程序计数器)。这种方法通过重叠连续指令的步骤来获得高性能。	\\[1ex]
\textbf{8)}	\\
补码除2的幂	\\
通过移位运算来实现对应值的除法	\\
$(x<0\ ?\ x+(1<<k)-1 : x) >> k$	\\[1ex]
\newpage
\noindent\textbf{9)}	\\
32位机器, typedef unsigned long long int uint64\_t; printf(``\%llu'',x)	\\
64位机器, typedef unsigned long int uint64\_t; printf(``\%lu'',x)	\\
\textbf{10)}	\\
C语言大数定义	\\
\#define max\_long 9223372036854775808llu	\\
\textbf{11)}
逆向工程循环	\\
我们描述 fact\_do 的过程对于逆向工程循环来说,是一个通用的策略。看看在循环之前如何初始化寄存器,在循环中如何更新和测试寄存器,以及在循环之后又如何使用寄存器。这些步骤中的每一步都提供了一个线索,纵使起来就可以解开谜团。做好准备,你会看到令人惊奇的变换,其中有些情况很明显是编译器能够优化代码,而有些情况很难理解编译器为什么要选用那些奇怪的策略。\red{根据我们的经验,GCC常常做的一些变换,非但不能带来性能好处,反而甚至可能降低代码性能。}	\\[2ex]
\textbf{\red{ 图灵机理论以及其它一些基础而重要的理论需要了解学习 }}	\\
\textbf{12)}	\\
简洁性与普适性是计算机系统追求的普适方向	\\
\textbf{13)}	\\
图灵机是对现实世界的高度抽象与建模,正因如此现代计算机的应用领域才会如此宽广,才能在各行各业具象化	\\
\textbf{14)}	\\
从汇编代码可以看出任何独立程序都有返回值\%rax	\\[2ex]
\textbf{15)}	\\
在计算机领域,由于信息在最底层,从物理上都是以二进制存储的,因经,结合二进制的信息特性与处理的位运算特性(与\&,或$|$,非!,异或\^{}),可以从二进制的处理上设计出一些十分高效的代码。	\\
例如:	\\
\newpage	
\begin{table}[ht]
\begin{tabular}{m{2em}m{2em}m{2em}l}
	\multicolumn{3}{l}{long fun\_a(unsigned long x)}	\\
\{	\\
	&	\multicolumn{3}{l}{long val = 0;}	\\
	&	\multicolumn{3}{l}{ while (x != 0) \{ } 	\\
	&	&	\multicolumn{2}{l}{val = val \^{} x;}	\\
	&	&	\multicolumn{2}{l}{x = x $>>$ 1;}	\\
	&	\multicolumn{3}{l}{ \} } 	\\
	&	\multicolumn{3}{l}{val = val \& 1;} 	\\
	&	\multicolumn{3}{l}{return val;} 	\\
\}	\\
\end{tabular}
\end{table}	
\noindent此函数统计参数x的二进制位表示中1的个数,若位表示中有奇数个1,则返回1;有偶数个1,则返回0。	\\[2ex]




\end{CJK}
\end{document}
