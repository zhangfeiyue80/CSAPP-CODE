\documentclass{article}
\usepackage{CJK}
\usepackage{type1cm}
\usepackage{color}
\usepackage{array}
\usepackage[margin=1cm,top=1cm,bottom=2cm]{geometry}
\begin{document}
\begin{CJK}{UTF8}{gkai}
\end{CJK}
\begin{CJK}{UTF8}{gbsn}
\end{CJK}
\begin{CJK}{UTF8}{uyhei}
%\textcolor[rgb]{1,0,0}{\begin{math} \end{math} \begin{math}			\end{math}}
%\centerline{\Huge{CSAPP Study Log}}		\\[3ex]
\fontsize{40pt}{40pt}\selectfont
\centerline{CSAPP Study Log}		
\fontsize{18pt}{18pt}\selectfont
\noindent\textbf{1)}			\\
$_w^u+x$无符号加法			\\[0.5ex]
表示与x相加后,取w位无符号整数		\\[0.5ex]
$_w^t+x$补码加法			\\[0.5ex]
$-_w^tx$表示x在$^t_w+$下的加法逆元	\\[0.5ex]
$-_w^ux$表示x在$_w^u+$下的加法逆元	\\[1ex]
\textbf{2)}
``试图最大化一段关键代码性能的程序员,通常会尝试源代码的各种形式''包括汇编	\\[1ex]
\textbf{3)}
精通细节是理解更深和更基本概念的先决条件	\\[1ex]
\textbf{4)}	\\
栈	stacks	\\[1ex]
栈指针	\%rsp	(stack pointer)	\\[1ex]
\textbf{5)}	\\
SAL(shift arithmetic left)	算术左移	\\
SHL(shift logical left)		逻辑左移	\\
SAR(shift arithmetic right)	自述右移	\\
SHR(shift logical right)	逻辑右移	\\[1ex]
\textbf{6)}	\\
补码	two's complement
\textbf{7)}	\\
处理器通过使用流水线(pipelining)来获得高性能,在流水线中,一条指令的处理要经过一系列的阶段,每个阶段执行所需操作的一小部分(例如,从内存取指令、确定指令类型、从内存读数据、执行算术去处、向内存写数据,以及更新程序计数器)。这种方法通过重叠连续指令的步骤来获得高性能。	\\[1ex]
\textbf{8)}	\\
补码除2的幂	\\
通过移位运算来实现对应值的除法	\\
$(x<0\ ?\ x+(1<<k)-1 : x) >> k$	\\





\end{CJK}
\end{document}
