\documentclass{article}
\usepackage{CJK}
\usepackage{type1cm}
\usepackage{color}
\usepackage{array}
\usepackage[margin=1cm,top=1cm,bottom=2cm]{geometry}
\newcommand{\blue}[1]{\textcolor{blue}{#1}}
\begin{document}
\begin{CJK}{UTF8}{uyhei}
\begin{CJK}{UTF8}{gkai}
\end{CJK}
\textcolor[rgb]{1,0,0}{\begin{math} \end{math} \begin{math}			\end{math}} \\[0cm]
\textbf{练习题1.1}\\[0.1cm]
A\\[0.1cm]
原先需要的时间25h \\[0.1cm]
改变的时间中,原先需要的时间15h \\[0.1cm]
改变的时间中,现在需要的时间10h \\[0.1cm]
节省的时间5h
加速比25/20=5/4=1.25 \\[0.3cm]
B\\[0.1cm]
原先需要的时间 (\begin{math} T_{old} \end{math} )\ 25h \\[0.3cm]
\begin{math} \frac{\textstyle T_{old}}{\textstyle T_{new}}=1.67 \end{math} \\[0.3cm]
\begin{math} T_{new}=\frac{T_{old}}{1.67}=\frac{25}{1.67}=14.97h \approx{}15h  \end{math} \\[0.3cm]
节省的时间10h \\[0.1cm]
原先通过蒙大拿需要的时间15h \\[0.1cm]
现在需要的时间5h \\[0.1cm]
现在的速度1500/5=300km/h \\[0.3cm]
\textbf{练习题1.2} \\[0.2cm]
\begin{math} \frac{\textstyle T_{old}}{\textstyle T_{new}}=2 \end{math}\\[0.2cm]
\begin{math} T_{new}=T_{old}/2 \end{math}\\[0.2cm]
节省的时间
\begin{math} T_{omit}=T_{old}-T_{new}=T_{old}/2 \end{math}\\[0.2cm]
\begin{math} k=\frac{0.8T_{old}}{0.8T_{old}-T_{omit}}=8/3=2.67 \end{math}\\[0.3cm]
\textbf{练习题2.1} \\[0.1cm]
\textbf{A} \ 001110011010011111111000 \\[0.1cm]
\textbf{B} \ C97B \\[0.1cm]
\textbf{C} \ 11010101111001001100 \\[0.1cm]
\textbf{D} \ 26E7B5 \\[0.7cm]
\textbf{练习题2.2}
\begin{tabbing}
xxxxx	\=	xxxxxxxxxxxxxx		\=	xxxxxxxxxxxxxxx	\kill
n	\>	$2^n$(十进制)  		\>	$2^n$(十六进制) 		\\
9	\>	512			\>	0x200				\\
19	\>	\textcolor{blue}{524288}	\>	\textcolor{blue}{0x80000}	\\
\textcolor{blue}{14}	\>	16384			\>	\textcolor{blue}{0x4000}				\\
\textcolor{blue}{16}	\>	\textcolor{blue}{65536} 			\>	0x10000				\\
17	\>	\textcolor{blue}{131072}	\>	\textcolor{blue}{0x20000}	\\
\textcolor{blue}{5}	\>	32	\>	\textcolor{blue}{0x20}		\\
\textcolor{blue}{7}	\>	\textcolor{blue}{128}			\>	0x80				
\end{tabbing} 
\ \\[0.2cm]
\textbf{练习题2.3}
\begin{tabbing}
	xxxxxxx	\=	xxxxxxxxxx	\=	xxxxxxxxx	\kill
	十进制	\>	二进制	\>	十六进制	\\
	0	\>	0000 0000	\>	0x00	\\
	167	\>	\textcolor{blue}{1010 0111} 	\>	\textcolor{blue}{0xA7} 	\\
	62	\>	\textcolor{blue}{11 1110}  	\>	\textcolor{blue}{0x3E} 	\\
	188	\>	\textcolor{blue}{1011 1100} 	\>	\textcolor{blue}{0xBC}			\\
	\textcolor{blue}{55}	\>	0011 0111	\>	\textcolor{blue}{0x37}	\\
	\textcolor{blue}{136}	\>	1000 1000	\>	\textcolor{blue}{0x88}	\\
	\textcolor{blue}{243}	\>	1111 0011	\>	\textcolor{blue}{0xF3}	\\
	\textcolor{blue}{82}	\>		\textcolor{blue}{0101 0010}	\>	0x52	\\
	\textcolor{blue}{172}	\>		\textcolor{blue}{1010 1100}	\>	0xAC	\\
	\textcolor{blue}{231}	\>		\textcolor{blue}{1110}	\>	0xE7	\\
\end{tabbing}
\textbf{练习题2.4} \\[0.2cm]
\textbf{A.} 0x5044 \\
\textbf{B.} 0x4FFc \\
\textbf{C.} 0x503c+64=0x503c+0x40=0x507c \\
\textbf{D.} 0xae
\ \\[0.5cm]
\textbf{练习题2.5} \\
\textbf{A.}
小端法:21
大端法:87 \\
\textbf{B.}
小端法: \\ 21 \\ 43 \\
大端法: \\ 87 \\ 65 \\
\textbf{C.}
小端法: \\ 21 \\ 43 \\ 65 \\
大端法: \\ 87 \\ 65 \\ 43 \\[0.5cm]
\textbf{练习题2.6}
\textbf{A.} \\ 0x00359141二进值表示0000 0000 0011 0101 1001 0001 0100 0001 \\
0x4A564504二进制表示0100 1010 0101 0110 0100 0101 0000 0100 \\
\textbf{B.}\ 有22位相匹配 \\
00000000001101011001000101000001 \\
\hspace*{1em}01001010010101100100010100000100 \\
\hspace*{5.5em}********************* \\
\textbf{C.}\ 串的开头和结尾不匹配 \\[3ex]
\textbf{练习题2.7} \\
输出结果为: \\
61 \\ 62 \\ 63 \\ 64 \\ 65 \\ 66 \\ 00 \\[3ex]
\textbf{练习题2.8} \\
\begin{tabbing}
	xxxxxxx		\= 	xxxxxxxxxxxx	\kill
	运算		\> 	[01101001]	\\
	a		\>	[01010101]	\\
	b		\>	[01010101]	\\
	~a		\>	\textcolor{blue}{[10010110]}	\\
	~b		\>	\textcolor{blue}{[10101010]}	\\
	a\&b		\>	\textcolor{blue}{[01000001]}	\\
	a$|$b		\>	\textcolor{blue}{[01111101]}	\\
	a\^{}b		\>	\textcolor{blue}{[00111100]}	\\
\end{tabbing}
\textbf{练习题2.9} \\
\textbf{A.}\\
\hspace*{1em} 黑色的补:		111	\\
\hspace*{1em} 蓝色的补:		110	\\
\hspace*{1em} 绿色的补:		101	\\
\hspace*{1em} 蓝绿色的补:	100	\\
\hspace*{1em} 红色的补:		011	\\
\hspace*{1em} 红紫色的补:	010	\\
\hspace*{1em} 黄色的补:		001	\\
\hspace*{1em} 白色的补:		000	\\
\textbf{B.}\\
\hspace*{1em}		蓝色 $|$ 绿色 = [001] $|$ [010] = [011] = 蓝绿色	\\
\hspace*{1em}		黄色 \& 蓝绿色 = [110] \& [011] = [010] = 绿色		\\
\hspace*{1em}		红色 \^{} 红紫色 = [100] \^{} [101] = [001] = 蓝色 \\[3ex]
\textbf{练习题2.10}
\begin{tabbing}
	xxxxxxxx	\=	xxxx	\=	xxxx	\kill
	步骤		\>	*x	\>	*y	\\
	初始		\>	a	\>	b	\\
	第1步		\>	\textcolor{blue}{a}	\>	\textcolor{blue}{a\^{}b}	\\
	第2步		\>	\textcolor{blue}{b}	\>	\textcolor{blue}{a\^{}b}	\\
	第3步		\>	\textcolor{blue}{b}	\>	\textcolor{blue}{a}	\\
\end{tabbing}
\textbf{练习题2.11}	\\
\textbf{A.}\  变量first和last的值都是k	\\
\textbf{B.}\ 此时first和last的值相同,都指向同一内存地址单元,第一步x\^{}y相当于此内存地址中的内容与自己异或,然后再将异或的值存放到此内存地址单元中,由a\^{}a=0,可知第一步后,内存单元中的值为0,此后相当于一直在运算0\^{}0,因此无论怎样,此内存地址中的值就一直为0了。\\
\textbf{C.}\ reverse\_array第四行``$<$=''改为``$<$'' \\[2ex]
\textbf{练习题2.12}	\\
\textbf{A.}\ x\&0xFF	\\
\textbf{B.}\ x\^{}(\~{}0xFF)	\\
\textbf{C.}\ x$|$0xFF	\\[2ex]
\textbf{练习题2.13}\\
bis(x,y)\\
bis(bic(x,y),bic(y,x))\\[3ex]
\textbf{练习题2.14}
\begin{tabbing}
	xxxxxxxxxx	\=	xxxxxxxxxxx	\=	xxxxxxxxxxxx	\=	xxxxxxxxxxx	\kill
	表达式		\>	值		\>	表达式	\>	值	\\[1ex]
	x \& y		\>	\textcolor{blue}{0x20}		\>	x \&\& y	\>	0x01	\\[1ex]
	x $|$ y		\>	\textcolor{blue}{0x7F}		\>	x $|$$|$ y	\>	\textcolor{blue}{0x01}		\\[1ex]
	\~{}x $|$ \~{}y	\>	\textcolor{blue}{0xDF}		\>	!x $|$$|$ !y	\>	\textcolor{blue}{0x00}	\\[1ex]
	x \& !y		\>	\textcolor{blue}{0x00}		\>	x \&\& \~{}y	\>	\textcolor{blue}{0x01}	\\[1ex]
\end{tabbing}
\textbf{练习题2.15}\\
!x\^{}y \\[3ex]
\textbf{练习题2.16}
\begin{tabbing}
	xxxxxxxxxxxxxxxxxxxxxxx	\=	xxxxxxxxxxxxxxxxxxxxxxx	\=	xxxxxxxxxxxxxxxxxxxxxxx	\=	xxxxxxxxxxxxxxxxxxxxxxx	\kill
	\hspace{4em}x		\>	\hspace{2em} x$<<$3		\>		\hspace{2em}x$>>$2(逻辑的)	\>		\hspace{2em} x$>>$2(算术的)	\\
	十六进制\hspace{2em}二进制	\>	二进制\hspace{2em}十六进制\>	二进制\hspace{2em}十六进制\>	二进制\hspace{2em}十六进制\\[2ex]
	0xC3	\hspace{2em}\textcolor{blue}{1100 0011}		\>	\textcolor{blue}{0001 1000}	\hspace{2em}\textcolor{blue}{0x18}	\>	\textcolor{blue}{0011 0000}	\hspace{2em} \textcolor{blue}{0x30}	\>	\textcolor{blue}{1111 0000}	\hspace{2em}\textcolor{blue}{0xF0}	\\
	0x75	\hspace{2em} \textcolor{blue}{0111 0101}		\>	\textcolor{blue}{1010 1000} \hspace{2em}\textcolor{blue}{0xA8}	\>	\textcolor{blue}{0001 1101}	\hspace{2em} \textcolor{blue}{0x1D}	\>	\textcolor{blue}{0001 1101}	\hspace{2em}\textcolor{blue}{0x1D}	\\
	0x87	\hspace{2em} \textcolor{blue}{1000 0111}		\>	\textcolor{blue}{0011 1000} \hspace{2em}\textcolor{blue}{0x38}	\>	\textcolor{blue}{0010 0001}	\hspace{2em} \textcolor{blue}{0x21}	\>	\textcolor{blue}{1110 0001}	\hspace{2em}\textcolor{blue}{0xE1}	\\
	0x66	\hspace{2em} \textcolor{blue}{0110 0110}		\>	\textcolor{blue}{0011 0000} \hspace{2em}\textcolor{blue}{0x30}	\>	\textcolor{blue}{0001 1001}	\hspace{2em} \textcolor{blue}{0x19}	\>	\textcolor{blue}{0001 1001}	\hspace{2em}\textcolor{blue}{0x19}	\\
\end{tabbing}
\textbf{练习题2.17}
\begin{tabbing}
	xxxxxxxxxxxxxxxxxxxxxxxxxxxxx	\=	xxxxxxxxxxxxxxxxxxxxxxxxxx	\=	xxxxxxxxxxxxxxx	\kill
	\hspace{5em}$\vec{x}$	\>	\hspace{2em}$B2U_4(\vec{x})$	\>		\hspace{3em}$B2T_4(\vec{x})$	\\
	十六进制 \hspace{2em}	二进制	\>		\>	\\[2ex]
	\hspace{1em}0xE	\hspace{3.5em}[1110]	\>	$2^3+2^2+2^1=14$	  \>		$-2^3+2^2+2^1=-2$	\\
	\hspace{1em}0x0 \hspace{3.7em}\textcolor{blue}{[0000]}	\>	\hspace{3em}\textcolor{blue}0		\>	\hspace{4em}\textcolor{blue}0	\\
	\hspace{1em}0x5	\hspace{3.7em}\textcolor{blue}{[0101]}	\>	\textcolor{blue}{$2^2+2^0=5$}		\>	\textcolor{blue}{$2^2+2^0=5$}		\\
	\hspace{1em}0x8	\hspace{3.7em}\textcolor{blue}{[1000]}	\>	\textcolor{blue}{$2^3=8$}		\>	\textcolor{blue}{$-2^3=-8$}		\\
	\hspace{1em}0xD	\hspace{3.4em}\textcolor{blue}{[1101]}	\>	\textcolor{blue}{$2^3+2^2+2^0=13$}	\>	\textcolor{blue}{$-2^3+2^2+2^0=-3$}	\\
	\hspace{1em}0xF	\hspace{3.5em}\textcolor{blue}{[1111]}	\>	\textcolor{blue}{$2^3+2^2+2^1+2^0=15$}	\>	\textcolor{blue}{$-2^3+2^2+2^1+2^0=-1$}	\\[1ex]
\end{tabbing}
\textbf{练习题2.18}\\[1ex]
\textbf{A.}\ 736 	\\
\textbf{B.}\ 88 	\\
\textbf{C.}\ 40		\\
\textbf{D.}\ 48		\\
\textbf{E.}\ 120	\\
\textbf{F.}\ -120	\\
\textbf{G.}\ 504	\\
\textbf{H.}\ -64	\\
\textbf{I.}\ 72		\\[5ex]
\textbf{练习题2.19}
\begin{tabbing}
	xxxxxxxx	\=	xxxxxxxxxxxxx		\kill
	x		\>	$T2U_4(x)$		\\
	-8		\>	\textcolor{blue}{8}	\\
	-3		\>	\textcolor{blue}{13}	\\
	-2		\>	\textcolor{blue}{14}	\\
	-1		\>	\textcolor{blue}{15}	\\
	0		\>	\textcolor{blue}{0}	\\
	5		\>	\textcolor{blue}{5}	\\
\end{tabbing}
\textbf{练习题2.21}
\begin{tabbing}
	xxxxxxxxxxxxxxxxxxxxxxxxxxxxxxxx	\=	xxxxxxxxxxxxxxxxxx		\=	xxxxxxx	\kill
	表达式					\>	类型			\>	求值				\\[0.5ex]
	-2147483647-1 == 2147483648U		\>	\textcolor{blue}{无符号}	\>	\textcolor{blue}1	\\
	-2147483647-1 $<$  2147483647		\>	\textcolor{blue}{有符号}	\>	\textcolor{blue}1	\\
	-2147483647-1U$<$  2147483647		\>	\textcolor{blue}{无符号}	\>	\textcolor{blue}0	\\
	-2147483647-1 $<$  -2147483647		\>	\textcolor{blue}{有符号}	\>	\textcolor{blue}1	\\
	-2147483647-1U$<$  -2147483647		\>	\textcolor{blue}{无符号}	\>	\textcolor{blue}1	\\
\end{tabbing}
\textbf{练习题2.22}\\
\textbf{A.}\ $-2^{3}+2+1=-5$ \\
\textbf{B.}\ $-2^{4}+2^3+2+1=-5$ \\
\textbf{C.}\ $-2^5+2^4+2^3+2+1=-5$ \\[3ex]
\textbf{练习题2.23}\\
\textbf{A.}
\begin{tabbing}
	xxxxxxxxxxxxxxxx	\=	xxxxxxxxxxxxxxxxx	\=	xxxxxxxxxxxxxxxxxxx	\kill
	w			\>	fun1(w)			\>	fun2(w)		\\
	0x00000076		\>	\textcolor{blue}{118}	\>	\textcolor{blue}{118}	\\
	0x87654321		\>	\textcolor{blue}{33}	\>	\textcolor{blue}{33}	\\
	0x000000C9		\>	\textcolor{blue}{201}	\>	\textcolor{blue}{-3639}	\\
	0xEDCBA987		\>	\textcolor{blue}{135}	\>	\textcolor{blue}{-3705}	\\
\end{tabbing}
\clearpage
\noindent\textbf{B.}\\ 
fun1将无符号整型逻辑左移24位,再逻辑右移24位,再转为有符号整形.\\
fun2将无符号整型转为有符号整型,再逻辑左移24位,再算术右移24位.\\[3ex]
\textbf{练习题2.24}
\begin{tabbing}
	xxxxxxxxxxxxxxxxxxx	\=	xxxxxxxxxxxxxxxxxxx			\=	xxxxxxxxxxxxxxxxxxx	\kill
	\hspace{2em}十六进制	\>	\hspace{2.5em}无符号			\>	\hspace{3em}补码			\\
	原始值\hspace{2em}截断值\>	原始值\hspace{2em}截断值\>	原始值\hspace{2em}截断值\\[1ex]
	0\hspace{5em}0		\>	0\hspace{5em}\textcolor{blue}{0}	\>	0\hspace{5em}\textcolor{blue}{0}	\\
	2\hspace{5em}2		\>	2\hspace{5em}\textcolor{blue}{2}	\>	2\hspace{5em}\textcolor{blue}{2}	\\
	9\hspace{5em}1		\>	9\hspace{5em}\textcolor{blue}{1}	\>	-7\hspace{5em}\textcolor{blue}{1}	\\
	B\hspace{5em}3		\>	11\hspace{5em}\textcolor{blue}{3}	\>	-5\hspace{5em}\textcolor{blue}{3}	\\
	F\hspace{5em}7		\>	15\hspace{5em}\textcolor{blue}{7}	\>	-1\hspace{5em}\textcolor{blue}{-1}	\\
\end{tabbing}
\textbf{练习题2.28}
\begin{tabbing}
	xxxxxxxxxxxx	\=	xxxxxxxxxxxx	\=	xxxxxxxxxxxx	\=	xxxxxxxxxxxx	\kill
			\>	x		\>			\>	$-^u_4x$	\\
	\hspace{1.5em}十六进制	\>	\hspace{1.5em}十进制		\>	\hspace{2.5em}十进制		\>	\hspace{2.5em}十六进制	\\
	\hspace{3em}0	\>	\hspace{2.5em}0	\>	\hspace{3.5em}0	\>	\hspace{4em}0	\\
	\hspace{3em}5	\>	\hspace{2.5em}5	\>	\hspace{3.5em}11\>	\hspace{4em}B	\\
	\hspace{3em}8	\>	\hspace{2.5em}8	\>	\hspace{3.5em}8	\>	\hspace{4em}8	\\
	\hspace{3em}D	\>	\hspace{2.5em}13\>	\hspace{3.5em}3	\>	\hspace{4em}3	\\
	\hspace{3em}F	\>	\hspace{2.5em}15\>	\hspace{3.5em}1	\>	\hspace{4em}1	\\
\end{tabbing}
\textbf{练习题2.29}
\begin{tabbing}
	xxxxxxxxxxxxxxx					\=	xxxxxxxxxxxxxxx					\=	xxxxxxxxxxxxxxx 					\=	xxxxxxxxxxxxxxx				\=	xxxxxxxxxxxxxxx				\kill
	\hspace{4em}x					\>	\hspace{4em}y					\>	\hspace{3em}x+y						\>	\hspace{2em}$x+^t_5y$			\>	\hspace{2em}情况			\\
	$[10100]$\hspace{2em}\textcolor{blue}{-12}	\>	$[10001]$\hspace{2em}\textcolor{blue}{-15}	\>	\textcolor{blue}{$[100101]$\hspace{2em}-27}		\>	\textcolor{blue}{[00101]\hspace{2em}5}	\>	\hspace{3em}\textcolor{blue}1		\\
	$[11000]$\hspace{2em}\textcolor{blue}{-8}	\>	$[11000]$\hspace{2em}\textcolor{blue}{-8}	\>	\textcolor{blue}{$[110000]$\hspace{2em}-16}		\>	\textcolor{blue}{[10000]\hspace{2em}-16}\>	\hspace{3em}\textcolor{blue}2		\\
	$[10111]$\hspace{2em}\textcolor{blue}{-9}	\>	$[01000]$\hspace{2em}\textcolor{blue}{8}	\>	\textcolor{blue}{$[111111]$\hspace{2em}-1}		\>	\textcolor{blue}{[11111]\hspace{2em}-1}	\>	\hspace{3em}\textcolor{blue}2		\\
	$[00010]$\hspace{2em}\textcolor{blue}{2}	\>	$[00101]$\hspace{2em}\textcolor{blue}{5}	\>	\textcolor{blue}{$[000111]$\hspace{2em}7}		\>	\textcolor{blue}{[00111]\hspace{2em}7}	\>	\hspace{3em}\textcolor{blue}3		\\
	$[01100]$\hspace{2em}\textcolor{blue}{12}	\>	$[00100]$\hspace{2em}\textcolor{blue}{4}	\>	\textcolor{blue}{$[010000]$\hspace{2em}16}		\>	\textcolor{blue}{[10000]\hspace{2em}-16}\>	\hspace{3em}\textcolor{blue}4		\\[1ex]
\end{tabbing}
\textbf{练习题2.31}	\\[2ex]
根椐补码加法的公式(2.13),可知当发生正溢出时,$sum=x+y-2^w$,此时$sum-x=y-2^w$,由于y是有符号整数,因此$y-2^w=y$;发生负溢出时,同理。因此,该同事的代码在发生溢出时,返回的值为1,显然,他的这段代码达不到我们的功能要求.	\\[3ex]
\textbf{练习题2.32}	\\[2ex]
$y=TMIN=-2^{31}$,x取任何值时;函数都会产生错误的结果	\\[3ex]
\textbf{练习题2.33}	\\[1ex]
\begin{tabular}[t]{cc|cc}
	\hline
	\multicolumn{2}{c}{$x$}				&	\multicolumn{2}{c}{$-_4^tx$}				\\
	\hline
	十六进制	&	十进制			&	十进制			&	十六进制		\\
	\hline
	0		&	\textcolor{blue}0	&	\textcolor{blue}0	&	\textcolor{blue}{0}	\\
	\hline
	5		&	\textcolor{blue}5	&	\textcolor{blue}{-5}	&	\textcolor{blue}{B}	\\
	\hline
	8		&	\textcolor{blue}{-8}	&	\textcolor{blue}{-8}	&	\textcolor{blue}{8}	\\
	\hline
	D		&	\textcolor{blue}{-3}	&	\textcolor{blue}{3}	&	\textcolor{blue}{3}	\\
	\hline
	F		&	\textcolor{blue}{-1}	&	\textcolor{blue}{1}	&	\textcolor{blue}{1}	\\
	\hline
\end{tabular}\\[1ex]
同一个数字的补码和无符号非产生的位模式相同。	\\[3ex]
\textbf{练习题2.34}	\\[1ex]
\begin{tabular}[t]{|c|c|c|c|c|}
	\hline	\\[-2ex]
	模式	&	x	&	y	&	$x\cdot y$	&	截断的$x\cdot y$	\\
	\hline	\\[-2ex]
	无符号	&	\textcolor{blue}{4}\ [100]	&	\textcolor{blue}{5}\ [101]	&	\textcolor{blue}{20}\ \textcolor{blue}{[010100]}		&	\textcolor{blue}{4}\ \textcolor{blue}{[100]}			\\[1ex]
	补码	&	\textcolor{blue}{-4}\ [100]	&	\textcolor{blue}{-3}\ [101]	&	\textcolor{blue}{12}\ \textcolor{blue}{[001100]}			&	\textcolor{blue}{-4}\ \textcolor{blue}{[100]}			\\
	\hline	\\[-2ex]
	无符号	&	\textcolor{blue}{2}\ [010]	&	\textcolor{blue}{7}\ [111]	&	\textcolor{blue}{14}\ \textcolor{blue}{[001110]}		&	\textcolor{blue}{6}\ \textcolor{blue}{[110]}			\\[1ex]
	补码	&	\textcolor{blue}{2}\ [010]	&	\textcolor{blue}{-1}\ [111]	&	\textcolor{blue}{-2}\ \textcolor{blue}{[111110]}		&	\textcolor{blue}{-2}\ \textcolor{blue}{[110]}			\\
	\hline	\\[-2ex]
	无符号	&	\textcolor{blue}{6}\ [110]	&	\textcolor{blue}{6}\ [110]	&	\textcolor{blue}{36}\ \textcolor{blue}{[100100]}		&	\textcolor{blue}{4}\ \textcolor{blue}{[100]}			\\[1ex]
	补码	&	\textcolor{blue}{-2}\ [110]	&	\textcolor{blue}{-2}\ [110]	&	\textcolor{blue}{4}\ \textcolor{blue}{[000100]}		&	\textcolor{blue}{-4}\ \textcolor{blue}{[100]}			\\
	\hline
\end{tabular}\\[3ex]
\textbf{练习题2.35}	\\[1ex]
1)由前面的叙述可以知道,$p=x*\ _w^ty=U2T_w((x\cdot y)mod\ 2^w)=(x\cdot y)mod\ 2^w-p_{w-1}2^w$,我们可以用64位的二进制数来表示$x\cdot y$,此时有	\\
\begin{math}
	x\cdot y=a_{63}2^{63}+a_{62}2^{62}+\dots+a_22^2+a_12+a_0	\\
	(x\cdot y)mod\ 2^w=(x\cdot y)mod\ 2^{32}=x\cdot y-a_{63}2^{63}-a_{62}2^{62}-\dots-a_w2^w
\end{math}	\\
就有	\\
$p=(x\cdot y)mod\ 2^w-p_{w-1}2^w=x\cdot y-a_{63}2^{63}-a_{62}2^{62}-\dots-a_w2^w-p_{w-1}2^w	\\
=x\cdot y-(a_{63}2^{31}+a_{62}2^{30}+\dots+a_w+p_{w-1})2^w	\\
=x\cdot y-t2^w$	\\
$-2^{w-1}\le p<2^{w-1}$,当$-2^{w-1}\le x\cdot y<2^{w-1}$时,计算不溢出	\\
$<1>$当$0\le x\cdot y<2^{w-1}$时,$p=(x\cdot y)mod\ 2^w-p_{w-1}2^w=x\cdot y$	\\
$<2>$当$-2^{w-1}\le x\cdot y<0$时,$p=(x\cdot y)mod\ 2^w-p_{w-1}2^w	\\
=(-2^{63}+2^{62}+\dots+2^w+2^{w-1}+a_{w-2}2^{w-2}+a_{w-3}2^{w-3}+\dots+a_22^2+a_12+a_0)mod\ 2^w-2^w	\\
=-2^{w-1}+a_{w-2}2^{w-2}+a_{w-3}2^{w-3}+\dots+a_22^2+a_12+a_0	\\
=x\cdot y$	\\
因此$x\cdot y=p+t2^w$,当且仅当p的计算溢出时,$t\neq 0$	\\
2)q是整数,因此q为p除以x取整,有$p=x\cdot q+x\cdot ((p/x)mod 1)$,于是就有$p=x\cdot q+r$,其中$|r|<|x|$			\\
3)由1,2可知,$p=x*\ _w^ty=(x\cdot y)mod\ 2^w-p_{w-1}2^w$,\ $p=x\cdot q+x\cdot ((p/x)mod 1)$			\\ 
有		\\
$(x\cdot y)mod\ 2^w-p_{w-1}2^w=x\cdot q+x\cdot ((p/x)mod 1)$	\\
$x\cdot y=p+t2^w$	\\
$x\cdot q=p-r$,当y=q时,有$p+t2^{w}=p-r$,有$t2^{w}=-r$,又因为$|r|<|x|\le|2^{w-1}|$,因此,r=t=0,因此有$x\cdot y=p$,计算不溢出.	\\[3ex]
\textbf{练习题2.37}	\\[1ex]
A.将乘法值赋值给无符号64位类型变量,避免了乘法溢出	\\
B.判断asize是否能用无符号32位来表示,如能表示调用malloc函数分配内存;如不能,给出提示,退出	\\[3ex]
\textbf{练习题2.38}	\\[1ex]
\begin{math}
	(a<<k)+b=a*2^k+b
\end{math}\\
当b=0时,\ $(a<<k)+b=a*2^k,k=0,1,2,3$	\\
此时,$(a<<k)+b$可表示a的8,4,2,1倍	\\[1ex]
当b=a时,\ $(a<<k)+b=a*2^k+a=a*(2^k+1),k=0,1,2,3$	\\
此时,$(a<<k)+b$可表示a的9,5,3,2倍	\\[3ex]
\textbf{练习题2.39}	\\[1ex]
当位位置n为最高有效位时,形式B变为如下形式:	\\
\begin{math}
	(x<<(n+1))-(x<<m)=-(x<<m)	\\[3ex]
\end{math}
\textbf{练习题2.40}	\\[1ex]
\begin{tabular}[t]{|c|c|c|c|}
	\hline		\\[-2ex]
	K	&	移位	&	加法/減法	&	表达式	\\
	\hline		\\[-2ex]
	6	&	2	&	1		&	$(x<<3)-(x<<1)$	\\
	\hline		\\[-2ex]
	31	&	1	&	1		&	$(x<<5)-x$	\\
	\hline		\\[-2ex]
	-6	&	2	&	1		&	$-(x<<3)-(x<<1)$	\\
	\hline		\\[-2ex]
	55	&	2	&	2		&	$(x<<6)-(x<<3)-x$	\\
	\hline
\end{tabular}	\\[3ex]
\textbf{练习题2.41}	\\[1ex]
形式A需要(n-m+1)次移位操作,需要(n-m)次加法操作	\\
形式B需要2次移位操作,需要1次减法操作	\\
可以看出形式B所做运算次数远小于形式A,因此编译器会使用形式B的方案	\\[3ex]
\textbf{练习题2.43}		\\[1ex]
M的值为31,N的值为8	\\[3ex]
\textbf{练习题2.44}	\\[1ex]
\textbf{A.}	\\
x是int型数值,因此$-2147483648\le x< 2147483648$	\\
当$x>0$时,$(x>0)||(x-1<0)$为真	\\
当$x\le 0$时,x的范围为$-2147483648<x\le 0$,有$-2147483648-1<x-1\le -1$, 因此$x-1<0$必成立	\\
综上,$(x>0)||(x-1<0)$必为真	\\
\textbf{B.} 	\\[1ex]
x是int型数值,因此$-2147483648\le x< 2147483648$	\\
\textbf{(1)} $(x\&7)!=7$为假	\\
此时应使$(x\&7)!=7$成立;	\\
当$(x\&7)=7$时,x的位表示为$\underbrace{***\dots***}_{29}111$(其中*表示0 or 1),因此$-2147483641=-x^{31}+7\le x< 2147483648$;	\\
要使$(x\&7)!=7$成立,应有$-2147483648\le x<-2147483641=-x^{31}+7$	\\
即当$-2147483648\le x<-2147483641=-x^{31}+7$时,$(x\&7)!=7$成立	\\
此时,x的位表示为:$\underbrace{***\dots***}_{29}111$,$x<<29$的位表示为$111\underbrace{000\dots00}_{29}$,因此,$x<<29<0$必为真	\\
因此,当$(x\&7)!=7$为假时,$(x<<29<0)$必为真,此时$(x\&7)!=7||(x<<29<0)$为真	\\
\textbf{(2)} 当$(x\&7)!=7$为真时,$(x\&7)!=7||(x<<29<0)$为真	\\
综上,$(x\&7)!=7||(x<<29<0)$必为真	\\
\textbf{C.}	\\
$x*\ _{32}^tx=U2T_{32}\left( (x*x)\ mod\ 2^{32} \right)$	\\
当$2^{ 31 }\le x*x<2^{ 32 }$时,$x*\ _{32}^tx<0$,此时$(x*x)\ge 0$为假	\\
\textbf{D.}	\\
\textbf{(1)} $x<0$为假,此时$x\ge 0$,$-x\le 0$必为真,此时$x<0 || -x\le 0$为真	\\
\textbf{(2)} $x<0$为真,此时$x<0 || -x\le 0$必为真	\\
综上,$x<0 || -x\le 0$为真	\\
\textbf{E.}	\\
\textbf{(1)} $x>0$为假,有$x\le 0$,则$-x\ge 0$成立,此时$x>0 ||-x\ge 0$为真	\\
\textbf{(2)} $x>0$为真,此时$x>0 ||-x\ge 0$为真	\\
综上,$x>0 ||-x\ge 0$为真	\\
\textbf{F.}	\\
由题可知
$x+\ _{32}^ty=U2T_{32}(ux+uy)$	\\
$ux+\ _{32}^uuy=( ux+uy ) mod\ 2^{32}$	\\
ux,uy是unsigned型,$0\le ux<2^{32}$,$0\le ux+uy<2^{33}$	\\
\textbf{(1)} 当$0\le ux+uy<2^{31}$时,$x+\ _{32}^ty=U2T_{32}(ux+uy)=ux+uy$,$ux+ _{32}^uuy=ux+uy$,此时$x+\ _{32}^ty=ux+\ _{32}^uuy$	\\
\textbf{(2)} 当$2^{31}\le ux+uy<2^{ 32 }$时,$x+\ _{32}^ty=U2T_{32}(ux+uy)=ux+uy-2^{32}$,$ux+\ _{32}^uuy=ux+uy$,此时,$x+\ _{32}^ty\ne ux+\ _{32}^uuy$	\\
\textbf{(3)} 当$2^{32}\le ux+uy<2^{33}$时,$x+\ _{32}^ty=U2T_{32}(ux+\ _{32}^uuy)=U2T_{32}(ux+uy-2^{32})$,$ux+\ _{32}^uuy=ux+uy-2^{32}$	\\
因为$0\le ux+uy-2^{32}<2^{32}$	\\
当$0\le ux+uy-2^{32}<2^{31}$时,$x+\ _{32}^ty=U2T_{32}(ux+uy-2^{32})=ux+uy-2^{32}$,$ux+\ _{32}^uuy=ux+uy-2^{32}$,此时$x+\ _{32}^ty=ux+\ _{32}^uuy$	\\
当$2^{31}\le ux+uy-2^{32}<2^{32}$时,$x+\ _{32}^ty=U2T_{32}(ux+uy-2^{32})=ux+uy-2^{32}-2^{32}=ux+uy-2^{33}$	\\
此时$2^{32}+2^{31}\le ux+uy<2^{33}$,$x+\ _{32}^ty\ne ux+\ _{32}^uuy$	\\
综上,当$2^{31}\le ux+uy<2^{32}$ and $2^{31}\le ux+uy<2^{33}$时,$x+\ _{32}^ty\ne ux+\ _{32}^uuy$,即$x+y == ux+uy$不成立	\\
\textbf{G.}	\\
\begin{math}
	x*\ _{32}^t\sim y+uy*\ _{32}^uux	\\
	=x*\ _{32}^t(-1-y)+ux*\ _{32}^uuy	\\
	=-x-x*\ _{32}^ty+ux*\ _{32}^uuy	\\
	=-x-U2T_{32}((x*y)mod\ 2^{32})+(ux*uy)mod\ 2^{32}
\end{math}\hspace{2em}由补码乘法和无符号乘法的位级等价性,有	\\
\begin{math}
	=-x-U2T_{32}\left ((ux*uy)mod\ 2^{32}\right )+(ux*uy)mod\ 2^{32}	\\
\end{math}
$0\le(ux*uy)mod\ 2^{32}<2^{32}$	\\
\textbf{(1)} 当$0\le(ux*uy)mod\ 2^{32}<2^{31}$时	\\
\begin{math}
	x*\ _{32}^t\sim y+ux*\ _{32}^uuy	\\
	=-x-U2T_{32}\left ((ux*uy)mod\ 2^{32}\right )+(ux*uy)mod\ 2^{32}	\\
	=-x-(ux*uy)mod\ 2^{32}+(ux*uy)mod\ 2^{32}	\\
	=-x
\end{math}	\\
\textbf{(2)} 当$2^{31}\le(ux*uy)mod\ 2^{32}<2^{32}$	\\
\begin{math}
	x*\ _{32}^t\sim y+ux*\ _{32}^uuy	\\
	=-x-U2T_{32}\left ((ux*uy)mod\ 2^{32}\right )+(ux*uy)mod\ 2^{32}	\\
	=-x-\left( (ux*uy)mod\ 2^{32}-2^{32}\right )+(ux*uy)mod\ 2^{32}	\\
	=-x-(ux*uy)mod\ 2^{32}+2^{32}+(ux*uy)mod\ 2^{32}	\\
	=-x+2^{32}	\\
\end{math}
综上,当$2^{31}\le(ux*uy)mod\ 2^{32}<2^{32}$时,$x*\sim y+uy*ux ==  -x$为假	\\[3ex]
\textbf{练习题2.45}	
\begin{table}[htbp]
	\centering
	\begin{tabular}{|l|c|r|}
		\hline
		小数值	&	二进制表示	&	十进制表示	\\
		\hline
		$\frac{1}{8}$	&	0.001	&	0.125	\\
		\hline
		$\frac{3}{4}$	&	\blue{0.11} &	\blue{0.75}		\\
		\hline
		$\frac{25}{16}$	&	\blue{1.1001}	&	\blue{1.5625}	\\
		\hline
		\blue{$\frac{43}{16}$}	&	10.1011	&	\blue{2.6875}	\\
		\hline
		\blue{ $\frac{9}{8}$ }		&	1.001	&	\blue{ 1.125 }\\
		\hline
		\blue{$\frac{47}{8}$}		&	\blue{ 101.01 }	&	5.875	\\
		\hline
		\blue{$\frac{51}{16}$}		&	\blue{11.0011}	&	3.1875	\\
		\hline
	\end{tabular}
\end{table}	\\
\textbf{练习题2.46}	\\
\textbf{A.} 	\\
\begin{math}
	0.1=0.000110011[0011][0011][0011][0011]\dots_2	\\
	  x=0.000110011 0011  0011  0011  00_2	\\
	  0.1-x=0.000000000 00000 00000 0000  0011 [0011]\dots_2
\end{math}	\\
\textbf{B.} 	\\
\begin{math}	
	x=0.000110011 0011  0011  0011  00_2=0.0999999046325684	\\
	0.1-x=0.1-0.0999999046325684=9.53674316045428\times10^{ -008 }
\end{math}	\\
\textbf{C.}	\\
\begin{math}	
	\Delta t=100*60*60*10*(0.1-x)=0.343322753776354(s)
\end{math}	\\
\textbf{D.}	\\
\begin{math}	
	\Delta x=2000*\Delta t=686.645507552708(m)
\end{math}	\\[3ex]
\newpage	
\renewcommand\arraystretch{1.5}
\noindent\textbf{练习题2.47}	
\begin{table}[ht]
\centering
\begin{tabular}{|m{5em}<{\centering}|m{5em}<{\centering}|m{5em}<{\centering}|m{5em}<{\centering}|m{5em}<{\centering}|m{5em}<{\centering}|m{5em}<{\centering}|m{5em}<{\centering}|m{5em}<{\centering}|}
	\hline
	位	&	e	&	E	&	$2^e$	&	$f$	&	M	&	$2^E\times M$	&	V	&	十进制	\\
	\hline
	0 00 00	&	0	&	0	&	1	&	0	&	0	&	$1\times 0$	&	0	&	0	\\
	\hline
	0 00 01	&	0	&	0	&	1	&	$\frac{1}{4}$	&	$\frac{1}{4}$	&	$1\times \frac{1}{4}$	&	$\frac{1}{4}$	&	0.25	\\
	\hline
	0 00 10	&	0	&	0	&	1	&	$\frac{1}{2}$	&	$\frac{1}{2}$	&	$1\times \frac{1}{2}$	&	$\frac{1}{2}$	&	0.5	\\
	\hline
	0 00 11	&	0	&	0	&	1	&	$\frac{3}{4}$	&	$\frac{3}{4}$	&	$1\times \frac{3}{4}$	&	$\frac{3}{4}$	&	0.75	\\
	\hline
	0 01 00	&	1	&	0	&	1	&	0	&	1	&	$1\times 1$	&	1	&	1	\\
	\hline
	0 01 01	&	1	&	0	&	1	&	$\frac{1}{4}$	&	$\frac{5}{4}$	&	$1\times \frac{5}{4}$	&	$\frac{5}{4}$	&	1.25	\\
	\hline
	0 01 10	&	1	&	0	&	1	&	$\frac{1}{2}$	&	$\frac{3}{2}$	&	$1\times \frac{3}{2}$	&	$\frac{3}{2}$	&	1.5	\\
	\hline
	0 01 11	&	1	&	0	&	1	&	$\frac{3}{4}$	&	$\frac{7}{4}$	&	$1\times \frac{7}{4}$	&	$\frac{7}{4}$	&	1.75	\\
	\hline
	0 10 00	&	2	&	1	&	2	&	0	&	1	&	$2\times 1$	&	2	&	2	\\
	\hline
	0 10 01	&	2	&	1	&	2	&	$\frac{1}{4}$	&	$\frac{5}{4}$	&	$2\times \frac{5}{4}$	&	$\frac{5}{2}$	&	2.5	\\
	\hline
	0 10 10	&	2	&	1	&	2	&	$\frac{1}{2}$	&	$\frac{3}{2}$	&	$2\times \frac{3}{2}$	&	3	&	3	\\
	\hline
	0 10 11	&	2	&	1	&	2	&	$\frac{3}{4}$	&	$\frac{7}{4}$	&	$2\times \frac{7}{4}$	&	$\frac{7}{2}$	&	3.5	\\
	\hline
	0 11 00	&	---	&	---	&	---	&	---	&	---	&	---	&	$\infty$	&	---	\\
	\hline
	0 11 01	&	---	&	---	&	---	&	---	&	---	&	---	&	NaN	&	---	\\
	\hline
	0 11 10	&	---	&	---	&	---	&	---	&	---	&	---	&	NaN	&	---	\\
	\hline
	0 11 11	&	---	&	---	&	---	&	---	&	---	&	---	&	NaN	&	---	\\
	\hline
\end{tabular}
\end{table}	\\
\textbf{练习题2.48}	\\[2ex]
3510593整数表示:	\\
$3510593=0x00359141=0011 0101 1001 0001 0100 0001_2$	\\[2ex]
3510593浮点表示:	\\
$3510593=1.1 0101 1001 0001 0100 0001\times 2^{21}$	\\
小数部分: 1 0101 1001 0001 0100 0001 00	\\[2ex]
21+127=148,$148=1001 0100_2$	\\
阶码部分: $10010100_2$	\\[2ex]
浮点表示为: $0 10010100 1 0101 1001 0001 0100 0001 00_2=0100 1010 0101 0110 0100 0101 0000 0100_2=0x4A564504$	\\
整数表示为:\hspace{4em}001\underline{101011001000101000001}	\\
浮点数表示为:01001010010101100100010100000100	\\
关系:整数有效位除去最高有效的一位的位表示与浮点表示小数部分的有效位表示相同	\\[3ex]
\textbf{练习题2.49}	\\[2ex]
\textbf{A.} 不能准确描述的最小正整数表示为:$2^{n+1}+1$	\\
\textbf{B.} 对于单精度格式,这个整数的值为:$2^{24}+1$	\\[3ex]
\textbf{练习题2.50}	\\[2ex]
\textbf{A.} $10.010_2\ 10.0$	\\
\textbf{B.} $10.011_2\ 10.1$	\\
\textbf{C.} $10.110_2\ 11.0$	\\
\textbf{D.} $11.001_2\ 11.0$	\\[2ex]
\textbf{练习题2.51}	\\[2ex]
\textbf{A.} $x'=0.1\approx 0.00011001100110011001101_2$	\\
\textbf{B.} $x'-0.1=0.100000023841858-0.1=0.000000023841858$	\\
\textbf{C.} $\Delta t=(x'-0.1)\times 100\times 60\times 60\times 10\ s=0.0858306888\ s$	\\
\textbf{D.} $\Delta x=\Delta t \cdot 2000\ m/s=171.6613776\ m$	\\[2ex]
\textbf{练习题2.52}
\begin{table}[htbp]
\centering
\begin{tabular}{|m{5em}<{\centering}|m{5em}<{\centering}|m{5em}<{\centering}|m{5em}<{\centering}|}
	\hline
	\multicolumn{2}{|c|}{格式A}	&	\multicolumn{2}{c|}{格式B}	\\
	\hline
	位	&	值	&	位	&	值	\\
	\hline
	011 0000	&	1	&	0111 000	&	1	\\
	\hline
	101 1110	&	15/2	&	1001 111	&	15/2	\\
	\hline
	010 1001	&	25/32	&	0110 100	&	3/4	\\
	\hline
	110 1111	&	31/2	&	1011 000	&	16	\\
	\hline
	000 0001	&	1/64	&	0001 000	&	1/64	\\
	\hline
\end{tabular}
\end{table}	\\[2ex]
\textbf{练习题2.53}	\\[2ex]
\#define POS\_INFINITY	1e400	\\
\#define NEG\_INFINITY	(-POS\_INFINITY)	\\
\#define NEG\_ZERO	(-1.0/POS\_INFINITY)	\\[3ex]
\textbf{练习题2.54}	\\[2ex]
\textbf{A.}	
真,因为double比int具有更大的精度和取值范围	\\
\textbf{B.}	
假,例如当x=TMAX时	\\
\textbf{C.}	
假,例如当d为最大非规格化数,即$d=0 00000000000 1111111111111111111111111111111111111111111111111111$	\\
\textbf{D.}	
真,因为double比float具有更大的精度和取值范围	\\
\textbf{E.}	
真,因为浮点数取负就是简单的对符号位取反	\\
\textbf{F.}	
真,在执行除法之前,分子和分母都会先转换成浮点数	\\
\textbf{G.}	
真,当d*d溢出为$\infty$时,仍然成立	\\
\textbf{H.}	
假,例如当f为1.0e20,d为1.0时,f+d舍入为f,f-f=0,因此表达式$(f+d)-f==d$为假	\\[2ex]
\textbf{练习题3.1}	\\
\begin{table}[htbp]
\centering
\begin{tabular}{|m{15em}<{\centering}|m{8em}<{\centering}|}
	\hline
	操作数	&	值	\\
	\hline
	\%rax	&	0x100	\\
	\hline
	0x104	&	0xAB	\\
	\hline
	\$0x108	&	0x108	\\
	\hline
	(\%rax)	&	0xFF	\\
	\hline
	4(\%rax)	&	0xAB	\\
	\hline
	9(\%rax,\%rdx)	&	0x11	\\
	\hline
	260(\%rcx,\%rdx)	&	0x13	\\
	\hline
	0xFC(,\%rcx,4)	&	0xFF	\\
	\hline
	(\%rax,\%rdx,4)	&	0x11	\\
	\hline
\end{tabular}
\end{table}	\\[2ex]
\newpage
\noindent\textbf{练习题3.2}	\\[2ex]
mov\underline{l} \%eax, (\%rsp)	\\
mov\underline{w} (\%rax), \%dx	\\
mov\underline{b} \$0xFF, \%bl	\\
mov\underline{b} (\%rsp,\%rdx,4), \%dl	\\
mov\underline{q} (\%rdx), \%rax	\\
mov\underline{w} \%dx, (\%rax)	\\[3ex]
\textbf{练习题3.3}	\\[2ex]
movb \$0xF, (\%ebx): 立即数0xF占用不够一个位,与mov类指令movb指定的位操作大小不符	\\
movl \%rax, (\%rsp): \%rax存放的是个占位4字大小的二进制数,movl指令指定的操作数的占位大小是双字,两者不符	\\
movw (\%rax), 4(\%rsp): mov类指令不能直接将位表示从内存的一个位置传到另一位置,需要借用到寄存器,通过两条指令来实现。	\\
movb \%al,\%sl: x86系列cpu中并没有名为sl的寄存器	\\
movq \%rax,\$0x123: 目的操作数不能是一个立即数	\\
movl \%eax,\%rdx: 源操作数是一个存储在寄存器中的空间大小为双字的数,目的操作数是一个能存储4字大小数的寄存器,两者不符	\\
movb \%si, 8(\%rbp): 寄存器si中存储的是一个字大小的数,movb指令指定的是对一个字节大小的数进行操作,两者不符	\\[3ex]
\textbf{练习题3.4}	\\[2ex]
\begin{table}[htbp]
\centering
\begin{tabular}{|m{8em}<{\centering}|m{8em}<{\centering}|m{12em}<{\centering}|}
	\hline
	src\_t	&	dest\_t	&	指令	\\
	\hline
	long	&	long	&	movq (\%rdi), \%rax	\\
		&		&	movq \%rax, (\%rsi)	\\
	char	&	int	&	\underline{movsbl (\%rdi), \%eax}	\\
	&	&	\underline{movl \%eax, (\%rsi)}	\\
	char	&	unsigned	&	\underline{movsbl (\%rdi), \%eax}	\\
	&	&	\underline{movl \%eax, (\%rsi)}	\\
	unsigned char	&	long	&	\underline{movzbq (\%rdi), \%rax}	\\
	&	&	\underline{movq \%rax, (\%rsi)}	\\
	int	&	char	&	\underline{movl (\%rdi), \%eax}	\\
	&	&	\underline{movb \%al, (\%rsi)}	\\
	unsigned	&	unsigned char	&	\underline{movl (\%rdi), \%eax}	\\
	&	&	\underline{movb \%al, (\%rsi)}	\\
	char	&	short	&	\underline{movsbw (\%rdi), \%ax}	\\
	&	&	\underline{movw \%ax, (\%rsi)}	\\
	\hline
\end{tabular}
\end{table}	\\[2ex]
\newpage
\noindent\textbf{练习题3.5}	\\[2ex]
void decode1(long *xp, long *yp, long *zp)	\\
\{	\\
	\hspace*{4em}	long x, y, z;	\\
	\hspace*{4em}	x = *xp;	\\
	\hspace*{4em}	y = *yp;	\\
	\hspace*{4em}	z = *zp;	\\
	\hspace*{4em}	*yp = x;	\\
	\hspace*{4em}	*zp = y;	\\
	\hspace*{4em}	*zp = z;	\\
	\hspace*{4em}	return z;	\\
\}	\\[3ex]
\textbf{练习题3.6}	\\
\begin{table}[htbp]
\centering
\begin{tabular}{|m{18em}<{\centering}|m{10em}<{\centering}|}
	\hline
	表达式	&	结果	\\
	\hline
	leaq 6(\%ax),\%rdx	&	$x\%2^4+6$	\\
	\hline
	leaq (\%rax,\%rcx),\%rdx	&	x+y	\\
	\hline
	leaq (\%rax,\%rcx,4),\%rdx	&	x+y*4	\\
	\hline
	leaq 7(\%rax,\%rax,8),\%rdx	&	9*x+7	\\
	\hline
	leaq 0xA(,\%rcx,4),\%rdx	&	4*y+0xA	\\
	\hline
	leaq 9(\%rax,\%rcx,2),\%rdx	&	x+2*y+9	\\
	\hline
\end{tabular}
\end{table}	\\[2ex]
\textbf{练习题3.7}	\\[2ex]
long scale2(long x,long y,long z)	\\
\{	\\
	long t = \underline{5x+2y+8z};	\\
	return t;	\\
\}	\\[3ex]
\textbf{练习题3.8}	\\[2ex]
\begin{table}[htbp]
\centering
\begin{tabular}{|m{18em}<{\centering}|m{10em}<{\centering}|m{10em}<{\centering}|}
	\hline
	指令	&	目的	&	值	\\
	\hline
	addq \%rcx,(\%rax)	&	0x100	&	0x100	\\
	\hline
	subq \%rdx,8(\%rax)	&	0x108	&	0xA8	\\
	\hline
	imulq \$16,(\%rax,\%rdx,8)	&	0x118	&	0x110	\\
	\hline
	incq 16(\%rax)	&	0x110	&	0x111	\\
	\hline
	decq \%rcx	&	rcx	&	0	\\
	\hline
	subq \%rdx,\%rax	&	rax	&	0xFD	\\
	\hline
\end{tabular}
\end{table}	\\[2ex]
\newpage
\noindent\textbf{练习题3.9}	\\[2ex]
shift\_left4\_rightn:	\\
\hspace*{1em} movq \%rdi, \%rax	\\
\hspace*{1em} shl \$4, \%rax	\\
\hspace*{1em} movl \%esi, \%ecx	\\
\hspace*{1em} sarq \%cl, \%rax	\\[3ex]
\textbf{练习题3.0}	\\[2ex]
long arith2(long x,long y,long z)	\\
\{	\\
\hspace*{2em}	long t1 = \underline{$x | y$};	\\
\hspace*{2em}	long t2 = \underline{$t1 >> 3$};	\\
\hspace*{2em}	long t3 = \underline{$\sim{t2}$};	\\
\hspace*{2em}	long t4 = \underline{z - t3};	\\
\hspace*{2em}	return t4;	\\
\}	\\[3ex]
\textbf{练习题3.11}	\\[2ex]
\textbf{A.} 将寄存器rdx所有位置0	\\
\textbf{B.} movq \$0, \%rdx	\\
\textbf{C.} 实现一个置0运算,用xor \%edx,\%edx 编码需2个字节;用mov \$0,\%rdx编码需5个字节。	\\[3ex]
\textbf{练习题3.12}	\\[2ex]
\blue{void uremdiv(long x, long y, long *qp, long *rp)}	\\
\blue{x in \%rdi, y in \%rsi, qp in \%rdx, rp in \%rcx}	\\
remdiv:	\\
\hspace*{2em}	movq	\%rdx, \%r8	\hspace*{2em}\blue{Copy qp}	\\
\hspace*{2em}	movq	\%rdi, \%rax	\hspace*{1.7em}\blue{Move x to lower 8 bytes of dividend}	\\
\hspace*{2em}	xorq	\%rdx, \%rdx	\hspace*{1.8em}\blue{Sign-extend to upper 8 bytes of dividend}	\\
\hspace*{2em}	divq	\%rsi		\hspace*{5.2em}\blue{Divide by y}	\\
\hspace*{2em}	movq 	\%rax, (\%r8)	\hspace*{1.3em}\blue{Store quotient at qp}	\\
\hspace*{2em}	movq	\%rdx, (\%rcx)	\hspace*{0.8em}\blue{Store remainder at rp}	\\
\hspace*{2em}	ret	\\[3ex]
\textbf{练习题 3.13}	\\[2ex]
\textbf{A.} data\_t是 int 数据类型,COMP是$<$	\\
\textbf{B.} data\_t是 short 数据类型,COMP是$>=$	\\
\textbf{C.} data\_t是 unsigned char 数据类型,COMP是$<=$	\\
\textbf{D.} data\_t是 long or unsigned long 数据类型,COMP是$!=$	\\[3ex]
\textbf{练习题3.14}	\\[2ex]
\textbf{A.} date\_t是 long 数据类型,COMP是$>=$	\\
\textbf{B.} date\_t是 short or unsigned short 据类型,COMP是$==$	\\
\textbf{C.} date\_t是 unsigned char 数据类型,COMP是$>$	\\
\textbf{D.} date\_t是 int or unsigned int 数据类型,COMP是$!=$	\\[3ex]
\textbf{练习题3.15}	\\[2ex]
\textbf{A.}	\\[-3ex]
\begin{table}[htbp]
\begin{tabular}{m{10em}m{10em}}
4003fa: 74 02	&	je \blue{4003fe}	\\
4003fc: ff d0	&	callq *\%rax	\\
\end{tabular}
\end{table}	\\
\newpage
\noindent\textbf{B.} 	\\[-3ex]
\begin{table}[htbp]
\begin{tabular}{m{10em}m{10em}}
40042f: 74 f4	&	je \blue{400425}	\\
400431: 5d	&	pop \%rbp	\\
\end{tabular}
\end{table}	\\
\textbf{C.} 	\\[-3ex]
\begin{table}[htbp]
\begin{tabular}{m{10em}m{10em}}
\blue{400423}: 77 02	&	ja 400547	\\
\blue{400425}: 5d	&	pop \%rbp	\\
\end{tabular}
\end{table}	\\
\textbf{D.} 	\\[-3ex]
\begin{table}[htbp]
\begin{tabular}{m{10em}m{10em}}
4005e8: e9 73 ff ff ff	&	jmpq \blue{400560}	\\
4005ed: 90	&	nop 	\\
\end{tabular}
\end{table}	\\
\textbf{练习题3.16}	\\[2ex]
\textbf{A.}	\\
void cond(long a, long *p)				\\
\{							\\
	\hspace*{2em}	if(p==0)			\\
	\hspace*{4em}		goto p\_not\_ez;	\\
	\hspace*{2em}	if(a $<=$ *p)			\\
	\hspace*{4em}		goto p\_not\_ez;	\\
	\hspace*{2em}	*p=a;				\\
	\hspace*{2em}	p\_not\_ez:			\\
\}							\\[2ex]
\textbf{B.}	\\
C语言代码中的if条件,包含两个判断,只有当p为1,且a$>$*p这两个条件都成立时,才会执行给定的语句。对于给定的两个条件,c能用一个表达式来实现,但汇编却一次只能实现一个判断,因此需要两个条件分支来实现两个判断。	\\[3ex]
\textbf{练习题3.17}	\\[2ex]
\textbf{A.}	\\
long gotodiff\_se(long x, long y)	\\
\{	\\
	\hspace*{2em}	long result;		\\
	\hspace*{2em}	if (x $<$ y)		\\
	\hspace*{4em}		goto x\_l\_y;	\\
	\hspace*{2em}	ge\_cnt++;		\\
	\hspace*{2em}	result = x - y;		\\
	\hspace*{2em}	return result;		\\
	\hspace*{1em} x\_l\_y:			\\
	\hspace*{2em}	lt\_cnt++;		\\
	\hspace*{2em}	result = y - x;		\\
	\hspace*{2em}	return result;		\\
\}	\\[1ex]
\textbf{B.}	\\
( 原先答案: jmp类指令不同的指令所需执行的操作数是不同的,选用jmp最少操作数的规则即可。 )	\\
在大多数情况下,可以在这两种方式中任意选择。但是原来的方法对常见的没有else语句的情况更好一些。对于这种情况,我们只用简单地将翻译规则修改如下:	\\
	\hspace*{2em}	t = test-expr;		\\
	\hspace*{2em}	if (!t)			\\
	\hspace*{4em}		goto done;	\\
	\hspace*{2em}	then-statement		\\
	\hspace*{1em} done:			\\
	基于这种替代规则的翻译更麻烦一些。		\\[3ex]
\textbf{练习题3.18}	\\
long test(long x, long y, long z)	\\
\{					\\
	\hspace*{2em}	long val = \underline{ x+y+z };	\\
	\hspace*{2em}	if (\underline{ -3$<$ x })		\\
	\hspace*{2em}	\{	\\
	\hspace*{4em}		if (\underline{ z$>=$y })	\\
	\hspace*{6em}			val = \underline{y*z};	\\
	\hspace*{4em}		else	\\
	\hspace*{6em}			val = \underline{x*y};	\\
	\hspace*{2em}	\}	\\
	\hspace*{2em}	else if (\underline{2$>$x})			\\
	\hspace*{4em}		val = \underline{x*z};		\\
	\hspace*{2em}	return val;						\\
\}					\\[3ex]
\textbf{练习题3.19}	\\[1ex]
\textbf{A.} 由题可知,$T_{OK}=16$, $T_{ran}=31$, 因此$T_{MP}=2(T_{ran}-T_{OK})=30$, 即预测错误处罚大约30个时钟周期。	\\[1ex]
\textbf{B.} $T_{wrong}=T_{OK}+T_{MP}=16+30=46$, 即当分支预测错误时,这个函数需要46个时钟周期。	\\[3ex]
\textbf{练习题3.20}	\\[1ex]
\textbf{A.} /。OP进行的是除法操作。	\\[1ex]
\textbf{B.} 	\\
\hspace*{2em}	\blue{long arith(long x)}	\\
\hspace*{2em}	\blue{x in \%rdi}		\\
	       arith:						\\
		\hspace*{2em}	leaq	7(\%rdi), \%rax		\hspace*{2em}	\blue{\%rax=7+x}	\\
		\hspace*{2em}	testq	\%rdi, \%rdi		\hspace*{2em}	\blue{\%rdi \& \%rdi}	\\
		\hspace*{2em}	cmovns	\%rdi, \%rax		\hspace*{2em}	\blue{\%rdi(x)是否为负。非负 \%rax=x; 负 \%rax=7+x}	\\
		\hspace*{2em}	sarq	\$3, \%rax		\hspace*{2em}	\blue{$\%rax=\%rax/2^3$}	\\
		\hspace*{2em}	ret				\\[3ex]
\textbf{练习题3.21}	\\[1ex]
long test(long x, long y)	\\
\{	\\
	\hspace*{2em}	long val = \underline{8*x};			\\
	\hspace*{2em}	if (\underline{y $>$ 0})				\\
	\hspace*{2em}	\{	\\
	\hspace*{4em}		if (\underline{x $>=$ y})			\\
	\hspace*{6em}			val = \underline{x \& y};	\\
	\hspace*{4em}		else						\\
	\hspace*{6em}			val = \underline{y - x};	\\
	\hspace*{2em}	\}	\\
	\hspace*{2em}	else if (\underline{y $<=$ -2})			\\
	\hspace*{4em}		val = \underline{x + y};		\\
	\hspace*{2em}	return val;	\\
\}	\\



\end{CJK}
\end{document}
